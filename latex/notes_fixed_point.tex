\documentclass{article}

%%%%%%%%%%%%%%%%%%%%%%%%%%%%%%%%%%%%%%%%%%%%%%%%%%%%%%%%
% Packages
%%%%%%%%%%%%%%%%%%%%%%%%%%%%%%%%%%%%%%%%%%%%%%%%%%%%%%%%

%\usepackage[utf8]{inputenc} % for overleaf/PLM
\usepackage[latin1]{inputenc} % averil local
\usepackage[T1]{fontenc} % hyphenation
\usepackage{fullpage} % DO NOT USE IN BEAMER
\usepackage[british,UKenglish,USenglish,american]{babel}
%\usepackage{appendix}
\usepackage{amssymb,amsmath,amsthm,enumerate}
\usepackage{mathtools} % coloneqq
%\usepackage{easybmat}
\usepackage{enumitem}
%\usepackage{tikz}
%\usepackage{caption}
\usepackage{float} % [H]
%\usepackage{bbold}
\usepackage[dvipsnames]{xcolor}
\usepackage{stmaryrd} % ll/rr brackets
%\usepackage[notcite, notref]{showkeys}
%\usepackage[tworuled,vlined,nofillcomment]{algorithm2e}
\usepackage[ruled,vlined]{algorithm2e}
%\usepackage{cases} % numbered lines in cases (numcases and subnumcases)
\usepackage[overload]{empheq} % source : https://tex.stackexchange.com/questions/31951/separate-labels-in-cases
\usepackage{caption} % to have subfigures
\usepackage{subcaption} % to have subfigures
\usepackage{cleveref} % \cref. 

%%%%%%%%%%%%%%%%%%%%%%%%%%%%%%%%%%%%%%%%%%%%%%%%%%%%%%%%
% Format
%%%%%%%%%%%%%%%%%%%%%%%%%%%%%%%%%%%%%%%%%%%%%%%%%%%%%%%%

\title{Notes on fixed-point procedure}
\author{} 
\date{}

\SetKwRepeat{Do}{do}{while} % for algorithm2e package, add do-while

% set dashes instead of bullets for item lists
\setlist[itemize,1]{label=$-$}
\setlist[itemize,2]{label=$-$}
\setlist[itemize,3]{label=$-$}

% remove unnecessary formatting of clever references
\crefdefaultlabelformat{(#2#1#3)}
\crefname{equation}{}{}
\crefname{figure}{figure}{Figure}

%%%%%%%%%%%%%%%%%%%%%%%%%%%%%%%%%%%%%%%%%%%%%%%%%%%%%%%%
% Theorems
%%%%%%%%%%%%%%%%%%%%%%%%%%%%%%%%%%%%%%%%%%%%%%%%%%%%%%%%

\newtheorem{proposition}{Proposition}[section]
\newtheorem{definition}{Definition}[section]
\newtheorem{theoreme}{Theorem}[section]
\newtheorem{remarque}{Remark}[section]
\newtheorem{lem}{Lemma}[section]
\numberwithin{equation}{section}

%%%%%%%%%%%%%%%%%%%%%%%%%%%%%%%%%%%%%%%%%%%%%%%%%%%%%%%%
% Commands
%%%%%%%%%%%%%%%%%%%%%%%%%%%%%%%%%%%%%%%%%%%%%%%%%%%%%%%%

\newcommand{\todo}[1]{{\color{red}\textbf{#1}}}
\newcommand{\vv}[1]{\begin{pmatrix} #1 \end{pmatrix}} % vector
\newcommand{\imh}{\textwidth} % meant to be redefined locally

\newcommand{\mysubeq}[2]{ % first argument : label, second : align content
	\begin{subequations}\label{#1}
		\begin{align}[left = {\empheqlbrace}]
			#2
		\end{align}
	\end{subequations}	
}
\newcommand{\mysubcaption}[1]{
	\vspace*{5pt}
	\begin{minipage}{0.8\linewidth}
		\begin{center}
			\footnotesize\emph{#1}
		\end{center}
	\end{minipage}
}
\newcommand{\myproof}[1]{
	\noindent \textbf{Demonstration}
	{\small	#1 \hfill \qedsymbol}
}
\newcommand{\intern}[1]{{\color{RoyalBlue} #1}} % will eventually be removed

% local vocabulary
\newcommand{\ve}{{\overline{v}_e}} % bound on the speed in the support of fe
\newcommand{\we}{{\underline{w}_e}} % transformation of \ve
\newcommand{\DomUpL}{{\mathcal{D}_1}} % domain over the critical char, x=0 side
\newcommand{\DomUpR}{{\mathcal{D}_2}} % domain over the critical char, x=1 side
\newcommand{\DomLow}{{\mathcal{D}_3}} % domain under the critical char + over -\ve
\newcommand{\IntUpL}{{\mathcal{I}_1}} % part of n_i on \DomUpL
\newcommand{\IntUpR}{{\mathcal{I}_2}} % part of n_i on \DomUpR
\newcommand{\IntLow}{{\mathcal{I}_3}} % part of n_i on \DomLow
\newcommand{\domfel}{{\underline{v}}} % lower bound on a segment on which fe >=  \minfe
\newcommand{\domfeu}{{\overline{v}}} % upper bound on a segment on which fe >=  \minfe
\newcommand{\minfe}{{\underline{c}}} % lower bound on f_e on a segment [-\domfeu, -\domfel]
\newcommand{\maxfe}{{\overline{c}}} % upper bound on f_e

%%%%%%%%%%%%%%%%%%%%%%%%%%%%%%%%%%%%%%%%%%%%%%%%%%%%%%%%
%%%%%%%%%%%%%%%%%%%%%%%%%%%%%%%%%%%%%%%%%%%%%%%%%%%%%%%%
%%%%%%%%%%%%%%%%%%%%%%%%%%%%%%%%%%%%%%%%%%%%%%%%%%%%%%%%

\begin{document}

\maketitle

%Let us define for all $0 < \alpha \leqslant \beta$ the set
%\begin{align*}
%	\mathcal{K}^{\alpha,\beta} \coloneqq \left\{\varphi \in \mathcal{C}^2\left([0,1], \mathbb{R}^{-}\right) \ |\ \varphi(0) = \varphi'(0) = 0, \quad - \beta \leqslant \varphi'' \leqslant - \alpha. \right\}
%\end{align*}
%
%\paragraph{Estimates on the set $\mathcal{K}$}
%
%Since $\varphi : [0,1] \mapsto \mathbb{R}^-$ is decreasing, its inverse $\varphi^{-1} : \mathbb{R}^- \mapsto [0,1]$ is well-defined. By integration and using $\left(\varphi^{-1}\right)' = (\varphi'\circ \varphi^{-1})^{-1}$, we have
%\mysubeq{eq:convex_bounds}{
%	- \beta x &\leqslant \varphi'(x) \leqslant - \alpha x \label{eq:convex_bounds_phiprime} \\
%	- \beta \frac{x^2}{2} &\leqslant \varphi(x) \leqslant - \alpha \frac{x^2}{2} \label{eq:convex_bounds_phi} \\
%	\sqrt{-\frac{2y}{\beta}} &\leqslant \varphi^{-1}(y) \leqslant \sqrt{-\frac{2y}{\alpha}} \label{eq:convex_bounds_phiinv} \\
%	\frac{-1}{\alpha\sqrt{-\frac{2y}{\beta}}} &\leqslant (\varphi^{-1})'(y) \leqslant \frac{-1}{\beta\sqrt{-\frac{2y}{\alpha}}} \label{eq:convex_bounds_phiinv} 
%}

\tableofcontents

% TO WRITE
% What are the characteristics equations
% The choice of the domain $v\leqslant0$ and $x\in[0,1]$
% Symmetries

\section{Notations and assumptions}

We suppose that 
\begin{enumerate}
\item $f_{e,b}$ satisfies the boundary condition, i.e. $f_{e,b}(v) = 0$ as soon as $\mathcal{L}_e(0,v) \leqslant \mathcal{L}_e(1,0)$.
\item $f_{e,b}$ is continuous. \todo{Lipschitz may be required?}
\item $f_{e,b}$ is bounded from above by a constant $\maxfe \geqslant 0$.
\item There exists $\minfe > 0$ and $0 \leqslant \domfel < \domfeu$ such that $f_{e,b}(v) \geqslant \minfe$ for all $v\in[-\domfeu,-\domfel]$.
\item The function $\varphi$ satisfies $\varphi(0)=\varphi'(0)=0$.
\item The function $\varphi$ is strongly concave, i.e. there exists $\alpha>0$ such that $\varphi((1-\tau) x + \tau y) \geqslant (1-\tau) \varphi(x)+\tau \varphi(y) + \frac{\alpha}{2} (1-\tau)\tau \left|x-y\right|^2$.
\item There exists $\beta > \alpha$ such that $\varphi(1) \geqslant - \beta$. \todo{Uniformly over the fixed-point iterations}
\end{enumerate}

We use the following notations: the Liapunov functions of the electrons and the ions are
\begin{align}\label{eq:def_Le}
	\mathcal{L}_e(x,v) \coloneqq \frac{v^2}{2} - \frac{1}{\mu} \varphi(x).
\end{align}
\begin{align}\label{eq:def_Li}
	\mathcal{L}_i(x,v) \coloneqq \frac{v^2}{2} + \varphi(x).
\end{align}

We define an upper bound $\ve$ over the velocities in the support of $f_e$, given as
\begin{align}\label{eq:def_ve}
	\mathcal{L}_e(0,\ve) \coloneqq \frac{1}{\mu} \beta \geqslant -\frac{1}{\mu}\varphi(1) = \mathcal{L}_e(1,0), \quad\text{i.e.}\quad \ve \coloneqq \sqrt{\frac{2}{\mu}\beta}.
\end{align}

For a given point $(x,v) \in [0,1]\times \mathbb{R}^{-}$, we denote by $(x_b, v_b)$ the intersection of the boundary $\left\{x=0\right\} \cup \left\{v=0\right\}$ with the ion characteristic issued from $(x,v)$. The values are given by
\begin{align}\label{eq:def_xb_vb}
	\vv{x_b(x,v) \\ v_b(x,v)} \coloneqq \vv{\varphi^{-1}\left(\min\left(0,\frac{v^2}{2} + \varphi(x)\right)\right) \\ - \sqrt{\max\left(0,v^2 + 2\varphi(x)\right)}}
%	\begin{cases}
%		\vv{\varphi^{-1}\left(\frac{v^2}{2} + \varphi(x)\right) \\ 0} & \text{if } \mathcal{L}_i(x,v) \leqslant 0, \\
%		\vv{0 \\ - \sqrt{v^2 + 2\varphi(x)}} & \text{if } \mathcal{L}_i(x,v) > 0.
%	\end{cases}
\end{align}


\section{Estimates}

\subsection{Useful elementary lemmas}

\begin{lem}\label{lem:int_1surR_rectangle}
	Let $\alpha>0$, $\overline{x}\geqslant0$ and $\overline{y}\geqslant0$. Then
	\begin{align*}
		\mathcal{I} \coloneqq \int_{x=0}^{\overline{x}} \int_{y=0}^{\overline{y}} \frac{1}{\left(x^2+\alpha y^2\right)^{1/2}} dy dx \leqslant 2 \frac{\sqrt{2\overline{x} \overline{y}}}{\alpha^{1/4}}.
	\end{align*}
\end{lem}

\myproof{
	By the change of variable $z=\sqrt{\alpha} y$, with $\overline{z} \coloneqq \sqrt{\alpha} \overline{y}$, we have
	\begin{align*}
		\mathcal{I} = \frac{1}{\sqrt{\alpha}} \int_{x=0}^{\overline{x}} \int_{z=0}^{\overline{z}} \frac{1}{\left(x^2+ z^2\right)^{1/2}} dz dx.
	\end{align*}
	Notice that $x+z \leqslant \sqrt{2} \left(x^2 + y^2\right)^{1/2}$. Then, 
	\begin{align*}
		\mathcal{I} 
		\leqslant \frac{1}{\sqrt{\alpha}} \int_{x=0}^{\overline{x}} \int_{z=0}^{\overline{z}} \frac{\sqrt{2}}{x+z} dz dx 
		= \sqrt{\frac{2}{\alpha}} \int_{x=0}^{\overline{x}} \ln\left(\frac{x+\overline{z}}{x}\right) dx
		= \sqrt{\frac{2}{\alpha}} \int_{x=0}^{\overline{x}} \ln\left(1 + \frac{\overline{z}}{x}\right) dx.
	\end{align*}
	Using $\ln\left(1+a\right) \leqslant \sqrt{a}$, we get
	\begin{align*}
		\mathcal{I} 
		\leqslant \sqrt{\frac{2}{\alpha}} \int_{x=0}^{\overline{x}} \sqrt{\frac{\overline{z}}{x}} dx 
		= \sqrt{\frac{2}{\alpha}} 2 \sqrt{\overline{x} \overline{z}}
		= 2 \frac{\sqrt{2 \overline{x} \overline{y}}}{\alpha^{1/4}}.
	\end{align*}
}

\intern{
\begin{remarque}[Exact value]
	Let $\overline{r} \coloneqq \sqrt{\overline{x}^2 + \overline{z}^2}$. Then 
	\begin{align*}
		\mathcal{I} = \overline{x} \ln\left(\frac{1 + \overline{z}/\overline{r}}{\overline{x}/\overline{r}}\right) + \overline{z} \ln\left(\frac{1 + \overline{x}/\overline{r}}{\overline{z}/\overline{r}}\right).
	\end{align*}
\end{remarque}
}

\subsection{Integrals along ion characteristics}

The estimates will rely on two particular cases, the we treat independently as lemmas. For a given $x$, we define $g_x : [0,x] \mapsto \mathbb{R}^{+}$ by
\begin{align*}
	\mathcal{L}_i(x,-g_x(y)) = \mathcal{L}_i(y,0), \quad \text{i.e.} \quad g_x(y) = \left(2\left(\varphi(y) - \varphi(x)\right)\right)^{1/2}.
\end{align*}

\begin{lem}\label{lem:upperbound_ni_endchar}
	Let $0\leqslant y < x \leqslant 1$. We have
	\begin{align*}
		\mathcal{I} \coloneqq \int_{v=-g_x(y)}^{0} \int_{z = x_b(x,v)}^{x} \frac{1}{\left(v^2-g_x^2(z)\right)^{1/2}} dz\,dv \leqslant 2 \sqrt{\frac{2}{\alpha}}\left(\varphi(y) - \varphi(x)\right)^{1/4} \sqrt{x-y}. 
	\end{align*}
\end{lem}

\myproof{
	Let us first use Fubini's theorem to switch the order of integration. The lower bound $x_b(x,v) \geqslant z$ becomes an upper bound $v \leqslant -g_x(z)$, and we have
	\begin{align*}
		\mathcal{I} 
		= \int_{z=y}^{x} \int_{v = -g_x(y)}^{-g_x(z)} \frac{1}{\left(v^2-g_x^2(z)\right)^{1/2}} dv\,dz 
		= \int_{z=y}^{x} \int_{v = g_x(z)}^{g_x(y)} \frac{1}{\left(v^2-g_x^2(z)\right)^{1/2}} dv\,dz.
	\end{align*} 
	With the change of variable $w = v - g_x(z)$, and using $\frac{d}{dw} \left[\sinh^{-1}\left(\sqrt{\frac{w}{a}}\right)\right] = \left(w^2 + 2aw\right)^{-1/2}$, we get
	\begin{align*}
		\mathcal{I} 
		= \int_{z=y}^{x} \int_{w = 0}^{g_x(y)-g_x(z)} \frac{1}{\left(w^2 + 2 w g_x(z)\right)^{1/2}} dw\,dz
		= \int_{z=y}^{x} \sinh^{-1}\left(\sqrt{\frac{g_x(y) -g_x(z)}{g_x(z)}}\right) dz.
	\end{align*}
	Using the coarse estimates $\sinh^{-1}(a) \leqslant a$ and $\sqrt{\frac{a-b}{b}} \leqslant \sqrt{\frac{a}{b}}$, we get
	\begin{align*}
		\mathcal{I} \leqslant \int_{z=y}^{x} \sqrt{\frac{g_x(y)}{g_x(z)}} dz = \int_{z=y}^{x} \left(\frac{\varphi(y) - \varphi(x)}{\varphi(z) - \varphi(x)}\right)^{1/4} dz.
	\end{align*}
	The assumption of strong convexity yields $\varphi(z) - \varphi(x) \geqslant - \varphi'(z) (x - z) + \frac{\alpha}{2} |x - z|^2 \geqslant \frac{\alpha}{2} (x - z)^2$, so that 
	\begin{align*}
		\mathcal{I} 
		\leqslant \sqrt{\frac{2}{\alpha}}\left(\varphi(y) - \varphi(x)\right)^{1/4} \int_{z=y}^{x} \frac{1}{\left(x - z\right)^{1/2}} dz 
		= 2 \sqrt{\frac{2}{\alpha}}\left(\varphi(y) - \varphi(x)\right)^{1/4} \sqrt{x-y}.
	\end{align*}
}

\begin{lem}\label{lem:upperbound_ni_beginchar}
	Let $0 \leqslant y \leqslant x \leqslant 1$, and $-v_0 < -g_x(y)$. We have
	\begin{align*}
		\mathcal{I} \coloneqq \int_{v=-v_0}^{-g_x(y)} \int_{z=y}^{1} \frac{1}{\left(v^2 + 2 \left(\varphi(x) - \varphi(z)\right)\right)^{1/2}} dz \, dv \leqslant \frac{2\sqrt{2(1-y)}}{\alpha^{1/4}} \left(v_0^2 + 2 \left(\varphi(x) - \varphi(y)\right)\right)^{1/4}.
	\end{align*}
\end{lem}

\myproof{
	We first shift the $v-$integration from the vertical line $z=x$ to $z=y$. Let $w = w(v)$ be such that
	\begin{align*}
		\mathcal{L}_i(y,w(v)) = \mathcal{L}_i(x,v), \quad \text{i.e.} \quad w(v) = -\left(v^2 + 2 \left(\varphi(x) - \varphi(y)\right)\right), \text{ and } dv = \frac{-w}{\left(w^2 + 2 \left(\varphi(y) - \varphi(x)\right)\right)^{1/2}}  dw.
	\end{align*}
	Then, defining $w_0 \coloneqq \left(v_0^2 + 2 \left(\varphi(x) - \varphi(y)\right)\right)^{1/2}$, and noticing that $w(-g_x(y)) = 0$, we get
	\begin{align*}
		\mathcal{I} = \int_{w=-w_0}^{0} \int_{z=y}^{1} \frac{1}{\left(w^2 + 2 \left(\varphi(y) - \varphi(z)\right)\right)^{1/2}} dz \frac{-w}{\left(w^2 + 2 \left(\varphi(y) - \varphi(x)\right)\right)^{1/2}} dw.
	\end{align*}
	Since $y \leqslant x$, we have $\varphi(y) \geqslant \varphi(x)$, and $\frac{-w}{\left(w^2 + 2 \left(\varphi(y) - \varphi(x)\right)\right)^{1/2}} \leqslant \frac{-w}{|w|} = 1$. By the strong convexity assumption, we have $\varphi(y) - \varphi(z) \geqslant - \varphi'(y) (z - y) + \frac{\alpha}{2} |z - y|^2 \geqslant \frac{\alpha}{2} (z - y)^2$, so that
	\begin{align*}
		\mathcal{I} \leqslant \int_{w=-w_0}^{0} \int_{z=y}^{1} \frac{1}{\left(w^2 + \alpha (z- y)^2\right)^{1/2}} dz dw = \int_{w=0}^{w_0} \int_{z=0}^{1-y} \frac{1}{\left(w^2 + \alpha z^2\right)^{1/2}} dz dw.
	\end{align*}
	Using \cref{lem:int_1surR_rectangle}, we conclude that 
	\begin{align*}
		\mathcal{I} \leqslant 2 \frac{\sqrt{2 w_0 (1-y)}}{\alpha^{1/4}} = \frac{2\sqrt{2(1-y)}}{\alpha^{1/4}} \left(v_0^2 + 2 \left(\varphi(x) - \varphi(y)\right)\right)^{1/4}.
	\end{align*}
}

\begin{proposition}
	The density $n_i$ is bounded. \todo{Be more precise}
\end{proposition}

\myproof{
	We use the symmetry of $f_i$ to write
	\begin{align*}
		n_i(x) = \int_{v=-\infty}^{\infty} f_i(x,v) dv = 2 \int_{v=-\infty}^{0} f_i(x_b(x,v), v_b(x,v)) dv = 2 \int_{v=-\infty}^{0} \int_{t=-\infty}^{0} f_e(x(t),v(t)) dt dv,
	\end{align*}
	where $(x(t),v(t))_{t\leqslant0}$ is the ion characteristic reaching $(x_b(x,v),v_b(x,v))$ at $t=0$. Notice that the lower bounds are artificial, since the characteristic enters the support of $f_e$ in finite time: we may use $v \geqslant -\ve$, and consider only times $t$ for which $x(t) \in [0,1]$. 
	
	We first reparametrize $(x(t),v(t))$ using the space variable. Define $z = x(t) \in[x_b,1]$, and observe that
	\begin{align*}
		dz = \dot{x}(t) dt = v(t) dt, \quad \text{with}\quad \mathcal{L}_i(z,v(t)) = \mathcal{L}_i(x,v) \quad\iff\quad v(t) = -\left(v^2 + 2\left(\varphi(x) - \varphi(z)\right)\right)^{1/2}.
	\end{align*}
	Then, the density rewrites
	\begin{align*}
		n_i(x) = 2 \int_{v=-\ve}^{0} \int_{z=x_b(x,v)}^{1} \frac{f_e\left(z, -\left(v^2 + 2\left(\varphi(x) - \varphi(z)\right)\right)^{1/2}\right)}{\left(v^2 + 2\left(\varphi(x) - \varphi(z)\right)\right)^{1/2}} \, dz dv.
	\end{align*}
	Let us show that $n_i$ is bounded. We use the coarse estimate $f_e \leqslant \maxfe$, and decompose the integral in three:
	\begin{align*}
		n_i(x) \leqslant 2 \maxfe \left[
		\underbrace{\int_{v=-g_x(0)}^{0} \int_{z=x_b(x,v)}^{x}}_{\IntUpL} + 
		\underbrace{\int_{v=-g_x(0)}^{0} \int_{z=x}^{1}}_{\IntUpR} + 
		\underbrace{\int_{v=-\ve}^{-g_x(0)} \int_{z=x_b(x,v)}^{1}}_{\IntLow} \right] 
		\frac{1}{\left(v^2 + 2\left(\varphi(x) - \varphi(z)\right)\right)^{1/2}} \, dz dv.
	\end{align*}
	The corresponding domains are represented \cref{fig:charmaps_domainmap}. 
	\begin{figure}
		\centering
		\includegraphics[width=0.5\linewidth]{images/fpcharmaps_domainmap}
		\caption{Decomposition of the integral defining $n_i$.}
		\mysubcaption{\todo{o boy so many things to say}}
		\label{fig:charmaps_domainmap}
	\end{figure}
	
	
	Notice that whenever $z \leqslant x$, we have $0 \geqslant 2\left(\varphi(x) - \varphi(z)\right) = - \left(2\left(\varphi(z) - \varphi(x)\right)\right)^{2/2} = - g_x^2(z)$. Then, the integral $\IntUpL$ may be bounded using \cref{lem:upperbound_ni_endchar} with $y=0$:
	\begin{align*}
		\IntUpL = \int_{v=-g_x(0)}^{0} \int_{z=x_b(x,v)}^{x} \frac{1}{\left(v^2 - g_x^2(z)\right)^{1/2}} \, dz dv 
		\leqslant 2 \sqrt{\frac{2}{\alpha}} \left(-\varphi(x)\right)^{1/4} \sqrt{x}
		\leqslant 2 \sqrt{\frac{2}{\alpha}} \left(-\varphi(1)\right)^{1/4}.
	\end{align*}
	
	We use \cref{lem:upperbound_ni_beginchar} to bound $\IntUpR$ and $\IntLow$. In the first case, we take $y=x$ and $v_0 = g_x(0)$, and notice that $-g_x(x)=0$. In the second case, we take $v_0 = \ve$ and $y=0$, and notice that on $v \leqslant -g_x(0)$, we have $x_b(x,v) = 0$ (the velocity is low enough so that the characteristic ends on $x_b=0$). This yields
	\begin{align*}
		\IntUpR \leqslant \frac{2\sqrt{2(1-x)}}{\alpha^{1/4}} \left(g_x^2(0)\right)^{1/4} \leqslant \frac{4}{\alpha^{1/4}} \left(-\varphi(1)\right)^{1/2} , \quad\text{and}\quad
		\IntLow \leqslant \frac{2\sqrt{2}}{\alpha^{1/4}} \left(\ve^2 + 2 \varphi(x)\right)^{1/4} \leqslant \frac{4}{\alpha^{1/4}} \left(\frac{\beta}{\mu}\right)^{1/4}.
	\end{align*}
}

\begin{proposition}
	The density $n_i$ is continuous. 
\end{proposition}

\myproof{
	Let $0 \leqslant y < x \leqslant 1$. For convenience, we represent $n_i(x)$ (resp. $n_i(y)$) as an integral with the artificial lower bound $-g_x(-\ve) \leqslant -\ve$ (resp. $-g_y(-\ve)$). Then
	\begin{align*}
		n_i(x) - n_i(y) 
		&= 2 \int_{v=-g_x(-\ve)}^0 f_i(x_b(x,v),v_b(x,v)) dv - 2 \int_{v=-g_y(-\ve)}^0 f_i(x_b(y,v),v_b(y,v)) dv \\
		&= 2 \underbrace{\left[\int_{v=-g_x(-\ve)}^{-g_x(y)} f_i(x_b(x,v),v_b(x,v)) dv - \int_{v=-g_y(-\ve)}^{0} f_i(x_b(y,v),v_b(y,v)) dv\right]}_{\eqqcolon\,\mathcal{I}^-}
		+ 2 \underbrace{\int_{v=-g_x(y)}^{0} f_i(x_b(x,v),v_b(x,v)) dv}_{\eqqcolon\,\mathcal{I}^+}.
	\end{align*}
}

Let us first focus on $\mathcal{I}^-$. On the first integral, we make the change of variable
\begin{align*}
	w = -\left(v^2 + 2 \left(\varphi(x) - \varphi(y)\right)\right)^{1/2} \quad v = -\left(w^2 + 2 \left(\varphi(y) - \varphi(w)\right)\right)^{1/2}.
\end{align*}
Since $\mathcal{L}_i(x,v) = \mathcal{L}_i(y,w)$, this yields $x_b(x,v) = x_b(y,w)$ and $v_b(x,v) = v_b(y,w)$. The bounds $v\in[-g_x(-\ve),-g_x(y)]$ are exactly transported to $w\in[-g_y(-\ve),0]$. Renaming $w$ in $v$, we get
\begin{align*}
	\mathcal{I}^- = \int_{v=-g_y(-\ve)}^{0} f_i(x_b(y,v),v_b(y,v)) \left(\frac{-v}{\left(v^2 + 2 \left(\varphi(y) - \varphi(v)\right)\right)^{1/2}} - 1\right) dv.
\end{align*}
Since $\varphi(y) \geqslant \varphi(x)$, the factor of $f_i$ is nonpositive, and so is $\mathcal{I}^-$. \todo{go on}

The second term $\mathcal{I}^+$ is clearly nonnegative, and may be adressed using our lemmas. Indeed, using the integral representation of $f_i(x_b,v_b)$ and the reparametrization by a space variable $z$, we have
\begin{align*}
	\mathcal{I}^+ 
	&= \int_{v=-g_x(y)}^{0} \left[\int_{z=x_b(x,v)}^{x} + \int_{z=x}^{1}\right] \frac{f_e(z,-\left(v^2+2\left(\varphi(x)-\varphi(z)\right)\right)^{1/2}))}{\left(v^2+2\left(\varphi(x)-\varphi(z)\right)\right)^{1/2}} \,dz dv \\
	&\leqslant \maxfe \int_{v=-g_x(y)}^{0} \int_{z=x_b(x,v)}^{x} \frac{1}{\left(v^2-g_x^2(z)\right)^{1/2}} \, dz dv + \maxfe \int_{v=-g_x(y)}^{0} \int_{z=x}^{1} \frac{1}{\left(v^2+2\left(\varphi(x)-\varphi(z)\right)\right)^{1/2}} \,dz dv \\
	&\leqslant 2\sqrt{\frac{2}{\alpha}} \left(\varphi(y)-\varphi(x)\right)^{1/4} \sqrt{x-y} + \frac{2\sqrt{2}}{\alpha^{1/4}} \left(2(\varphi(y)-\varphi(x))\right)^{1/2},
\end{align*}
where we used \cref{lem:upperbound_ni_endchar} for the first term, and \cref{lem:upperbound_ni_beginchar} for the second term (with $y=x$ and $v_0 = -g_x(y)$ under the notations of the lemma).


%%%%%%%%%%%%%%%%%%%%%%%%%%%%%%%%%%%%%%%%%%%%%%%%%%%%%%%%%%%%%%
%%%%%%%%%%%%%%%%%%%%%%%%%%%%%%%%%%%%%%%%%%%%%%%%%%%%%%%%%%%%%%
%%%%%%%%%%%%%%%%%%%%%%%%%%%%%%%%%%%%%%%%%%%%%%%%%%%%%%%%%%%%%%

\if{oldies}

\subsection{Upper bounds on the densities}

We want to obtain estimates on $n_i - n_e$. We make the following assumptions:
\begin{itemize}
\item The electron density $f_e$ satisfies the boundary condition, and is bounded by a constant $\maxfe \geqslant 0$. 
\item The potential $\varphi$ is strongly concave, i.e. there exists $\alpha>0$ such that $\varphi''(x) \leqslant - \alpha$ uniformly over $x\in[0,1]$.
\item We have $\varphi(0) = \varphi'(0) = 0$.
\end{itemize}
The assumptions on $\varphi$ yield that
\begin{align}\label{eq:phi_concave_consequences}
	\varphi'(x) \leqslant- \alpha x, \quad \varphi(x) \leqslant - \alpha \frac{x^2}{2}.
\end{align}

Let us first focus on $n_e(x)$. The characteristics of the electron density are the level lines of
\begin{align}\label{eq:def_Le}
	\mathcal{L}_e(x,v) \coloneqq \frac{v^2}{2} - \frac{1}{\mu} \varphi(x).
\end{align}
Since $\varphi$ is strongly concave, these curves are closed. Since $f_e$ satisfies the homogeneous boundary condition, its support is embedded in $\left\{(x,v) \ |\ \frac{v^2}{2} - \frac{1}{\mu} \varphi(x) \leqslant \frac{0^2}{2} - \frac{1}{\mu} \varphi(1)\right\}$. In particular, we denote by $\ve$ the extremal speed of the support, given by
\begin{align}\label{eq:def_ve}
	\ve \coloneqq \sqrt{-\frac{2}{\mu} \varphi(1)}.
\end{align}
We can roughly majorize
\begin{align*}
	n_e(x) = \int_{v\in\mathbb{R}} f_e(x,v) dv \leqslant \int_{v=-\ve}^{\ve} \maxfe dv = 2 \maxfe \ve \leqslant 2 \maxfe \sqrt{-\frac{2}{\mu} \varphi(1)}.
\end{align*}

The estimates on $n_i$ are slightly more technical. Let the ion Lyapunov function be defined as
\begin{align}\label{eq:def_Li}
	\mathcal{L}_i(x,v) \coloneqq \frac{v^2}{2} + \varphi(x).
\end{align}
In the sequel, we will heavily rely on the level lines of $\mathcal{L}_i$ to partition the space. We distinguish the \emph{critical characteristic} as the curve $\left\{\mathcal{L}_i = 0\right\}$.
Let $x\in[0,1]$ and $v \in \mathbb{R}_{-}$. We denote by $(x_b(x,v),v_b(x,v))$ the intersection of the boundary $\left\{x=0\right\}\cap\left\{v=0\right\}$ with the characteristic issued from $(x,v)$, equal to
\begin{align*}
	\vv{x_b(x,v) \\ v_b(x,v)} \coloneqq 
	\begin{cases}
		\vv{\varphi^{-1}\left(\frac{v^2}{2} + \varphi(x)\right) \\ 0} & \text{if } \mathcal{L}_i(x,v) \leqslant 0, \\
		\vv{0 \\ - \sqrt{\frac{v^2}{2} + \varphi(x)}} & \text{if } \mathcal{L}_i(x,v) > 0.
	\end{cases}
\end{align*}
In the following paragraph, we use $(x(t),v(t))_{t\leqslant0}$ to denote the characteristic reaching $(x_b(x,v),v_b(x,v))$ at $t=0$. 
We use the symmetry of $f_i$ to write 
\begin{align*}
	n_i(x) = 2 \int_{v\in\mathbb{R}^{-}} f_i(x_b(x,v), v_b(x,v)) dv = 2 \int_{v\in\mathbb{R}^{-}}\int_{t=-\infty}^{0} f_e(x(t),v(t)) dt dv.
\end{align*}
The lower bound $t \to -\infty$ is artificial, since the characteristic exits the support of $f_e$ in finite time. We will split the double integral in three domains:
\begin{enumerate}
\item $\DomUpL$ will be $\left\{(v,t) \in \mathbb{R}_{-}^2 \ |\ \mathcal{L}_i(x,v) \leqslant 0 \text{ and } x(t) \leqslant x \right\}$. This is the region contained between the $x-$axis, the critical characteristic and the vertical line going through $x$.
\item $\DomUpR$ is $\left\{(v,t) \in \mathbb{R}_{-}^2 \ |\ \mathcal{L}_i(x,v) \leqslant 0 \text{ and } x < x(t) \leqslant 1 \right\}$. It cover the part of the domain intersecting $\left\{\mathcal{L}_i \leqslant 0\right\} \setminus \DomUpL$.
\item $\DomLow$ is defined by $\left\{(v,t) \in \mathbb{R}_{-}^2 \ |\ \mathcal{L}_i(x,v) > 0 \text{ and } -\ve \leqslant v(t) \right\}$. 
\end{enumerate}
The corresponding phase space decomposition is represented in \cref{fig:charmaps_domainmap}.

\begin{figure}
	\centering
	\includegraphics[width=0.5\linewidth]{images/fpcharmaps_domainmap}
	\caption{Representation in the phase space $(x,v)$ of the decomposition of $(x,t)\in\mathbb{R}_-^2$ in $\DomUpL\cup\DomUpR \cup \DomLow$.}
	\label{fig:charmaps_domainmap}
\end{figure}

Since $f_e(x(t),v(t))$ vanishes outside $\DomUpL \cup \DomUpR \cup \DomLow$, we may exactly decompose $n_i$ in 
\begin{align*}
	\frac{n_i(x)}{2} = \underbrace{\iint_{(v,t) \in \DomUpL} f_e(x(t),v(t)) dt dv}_{\IntUpL} + \underbrace{\iint_{(v,t) \in \DomUpR} f_e(x(t),v(t)) dt dv}_{\IntUpR} + \underbrace{\iint_{(v,t) \in \DomLow} f_e(x(t),v(t)) dt dv}_{\IntLow} 
\end{align*}

Each term will be bound separately. 

\paragraph{Bound on $\IntUpL$}

Let $v_0(x) \coloneqq \sqrt{-2\varphi(x)}$ be the velocity such that $(x, -v_0(x))$ belongs to the critical characteristic. 
The characteristics in the domain $\DomUpL$ are joining points $(x,v)$, with $v \in [-v_0(x),0]$, with points $(x_b(x,v), 0)$. Therefore, we may use the reparametrization 
\begin{align*}
	y = x(t), \quad dy = \dot{x}(t) dt = v(t) dt = - \left(v^2 + 2 \left(\varphi(x) - \varphi(x(t))\right)\right)^{1/2} dt = - \left(v^2 + 2 \left(\varphi(x) - \varphi(y)\right)\right)^{1/2} dt	
\end{align*}
The integral $\IntUpL$ becomes
\begin{align*}
	\IntUpL = \int_{v=-v_0(x)}^0 \int_{y=x}^{x_b(x,v)} f_e(y,- \left(v^2 + 2 \left(\varphi(x) - \varphi(y)\right)\right)^{1/2}) \frac{-1}{\left(v^2 + 2 \left(\varphi(x) - \varphi(y)\right)\right)^{1/2}} dy dv.
\end{align*}
By exchanging the bounds of the integrals along $y$, and using $f_e \leqslant \maxfe$, we get
\begin{align*}
	\IntUpL \leqslant \maxfe \int_{v=-v_0(x)}^0 \int_{y=x_b(x,v)}^{x} \frac{1}{\left(v^2 + 2 \left(\varphi(x) - \varphi(y)\right)\right)^{1/2}} dy dv.
\end{align*}
In order to use the explicit $v^2$, we use Fubini theorem to switch the order of integration (since everything is positive). To do this, we write
\begin{align*}
	\begin{cases}
		- v_0(x) \leqslant v \leqslant 0 \\
		x_b(x,v) \leqslant y \leqslant x 
	\end{cases}
	\quad \iff \quad 
	\begin{cases}
		0 \leqslant y \leqslant x \\
		- v_0(x) \leqslant v \leqslant - g_x(y)
	\end{cases}
\end{align*}
where $y = x_b(x,v) = \varphi^{-1}\left(\frac{v^2}{2} + \varphi(x)\right)$ is equivalent to $v = - g_x(y) \coloneqq - \left(2\left(\varphi(y) - \varphi(x)\right)\right)^{1/2}$. In the sequel, we drop the $x$ and simply write $g(y)$. The function $g : [0,x] \mapsto \mathbb{R}^{+}$ is well-defined, since $y \leqslant x \implies \varphi(y) \geqslant \varphi(x)$. By the assumption of strong concavity of $\varphi$, $g$ is positive whenever $y < x$. Finally, using that $v_0(x) = g(0)$, may now write 
\begin{align*}
	\IntUpL \leqslant \maxfe \int_{y=0}^{x} \int_{v=-v_0(x)}^{-g(y)}  \frac{1}{\left(v^2 - g^2(y)\right)^{1/2}} dv dy \underset{\text{with }w\coloneqq g(y)-v}{=} \maxfe \int_{y=0}^{x} \int_{w=0}^{g(0)-g(y)}  \frac{1}{\left(w^2 + 2 w g(y)\right)^{1/2}} dv dy.
\end{align*}
By integration with $\frac{d}{dy} [\sinh^{-1}\left(\sqrt{\frac{y}{a}}\right)] = \frac{1}{(y^2 + 2 a y)^{1/2}}$, we obtain
\begin{align*}
	\IntUpL \leqslant \maxfe \int_{y=0}^{x} \left[\sinh^{-1}\left(\sqrt{\frac{v}{g(y)}}\right)\right]^{g(0)-g(y)}_{0} dv = \maxfe \int_{y=0}^{x} \sinh^{-1}\left(\sqrt{\frac{g(0)-g(y)}{g(y)}}\right) dv.
\end{align*}
We use the coarse estimates $\sinh^{-1}(z) \leqslant z$ and $\sqrt{\frac{a-b}{b}} \leqslant \sqrt{\frac{a}{b}}$ to reduce the expression to
\begin{align*}
	\IntUpL \leqslant \maxfe \sqrt{g(0)} \int_{y=0}^{x} \frac{1}{\sqrt{g(y)}} dv = \maxfe \sqrt{g(0)} \int_{y=0}^{x} \frac{1}{\left(2\left(\varphi(y) - \varphi(x)\right)\right)^{1/4}} dv.
\end{align*}
Using the strong concavity of $\varphi$, and the sign $-\varphi'(y) \geqslant 0$, we get
\begin{align*}
	\varphi(y) - \varphi(x) \geqslant \frac{\alpha}{2} |x - y|^2 - \varphi'(y)(x-y) \geqslant \frac{\alpha}{2} (x - y)^2.
\end{align*}
With this, we may finally write
\begin{align*}
	\IntUpL \leqslant \maxfe \sqrt{g(0)} \int_{y=0}^{x} \frac{1}{\alpha^{1/4}\left(x-y\right)^{1/2}} dv = \frac{2 \maxfe}{\alpha^{1/4}} \sqrt{g(0) x} = \frac{2 \maxfe}{\alpha^{1/4}} \sqrt{\sqrt{-2\varphi(x)} x} \leqslant \frac{2 \maxfe}{\alpha^{1/4}} \left(-2\varphi(1)\right)^{1/4}.
\end{align*}

\paragraph{Bound on $\IntUpR$}

We use the same reparametrization as for $\IntUpR$, but with $y \in [x,1]$, to obtain
\begin{align*}
	\IntUpR \leqslant \maxfe \int_{v=-v_0(x)}^{0} \int_{y=x}^{1} \frac{1}{\left(v^2 + 2(\varphi(x) - \varphi(y))\right)^{1/2}} dy dv.
\end{align*}
On $y \geqslant x$, we may directly use
\begin{align*}
	\varphi(x) - \varphi(y) \geqslant \frac{\alpha}{2} |y - x|^2 - \varphi'(x) (y-x) \geqslant \frac{\alpha}{2} (y - x)^2
\end{align*}
to get
\begin{align*}
	\IntUpR \leqslant \maxfe \int_{v=-v_0(x)}^{0} \int_{y=x}^{1} \frac{1}{\left(v^2 + \alpha(y-x)^2\right)^{1/2}} dy dv \leqslant \maxfe \max\left(1,\frac{1}{\sqrt{\alpha}}\right) \int_{v=0}^{v_0(x)} \int_{z=0}^{1-x} \frac{1}{\left(v^2 + z^2\right)^{1/2}} dy dv.
\end{align*}
By switching to polar coordinates over the (larger) domain $(\theta,r) \in [0,\frac{\pi}{2}] \times [0,\sqrt{v_0(x)^2 + (1-x)^2}]$, we conclude to 
\begin{align*}
	\IntUpR \leqslant \maxfe \max\left(1,\frac{1}{\sqrt{\alpha}}\right) \frac{\pi}{2} \sqrt{v_0(x)^2 + (1-x)^2} \leqslant \maxfe \max\left(1,\frac{1}{\sqrt{\alpha}}\right) \frac{\pi}{2} \sqrt{1 - 2\varphi(1)}.
\end{align*} 

\paragraph{Bound on $\IntLow$}

The domain $\DomLow$ is covered by characteristics linking $x=1$ to $x=0$. We may use the same reparametrization within fixed bounds over $y$:
\begin{align*}
	\IntLow \leqslant \maxfe \int_{v=-\ve}^{-v_0(x)} \int_{y=0}^{1} \frac{1}{\left(v^2 + 2 \left(\varphi(x) - \varphi(y)\right)\right)^{1/2}} dy dv = \maxfe \int_{y=0}^{1} \int_{v=-\ve}^{-v_0(x)} \frac{1}{\left(v^2 + 2 \left(\varphi(x) - \varphi(y)\right)\right)^{1/2}} dy dv.
\end{align*}
We will use the same argument as for $\IntUpR$, but with characteristics ending on $x=0$ instead of $x=x$. Our first step is then to replace $v$ by $w$ the velocity at $x=0$, defined by $\frac{v^2}{2} + \varphi(x) = \frac{w^2}{2} + 0$. We have
\begin{align*}
	 w = - \left(v^2 + 2 \varphi(x)\right)^{1/2} \in [-\we, 0], \quad v = - \left(v^2 - 2 \varphi(x)\right)^{1/2}, \quad dv = \frac{-w}{\left(w^2-2\varphi(x)\right)^{1/2}} dw,
\end{align*}
where $\we \coloneqq \left(\ve^2 + 2 \varphi(x)\right)^{1/2} \leqslant \ve \leqslant \sqrt{-2\varphi(1)}$. Using the estimate $-\varphi(y) \geqslant \alpha \frac{y^2}{2}$, we conclude similarly that
\begin{align*}
	\IntLow 
	\leqslant \maxfe \int_{y=0}^{1} \int_{w=-\we}^{0} \frac{1}{\left(w^2 - 2\varphi(y)\right)^{1/2}} dy dv 
	= \maxfe \int_{y=0}^{1} \int_{w=0}^{\we} \frac{1}{\left(w^2 + \alpha y^2\right)^{1/2}} dy dv
	\leqslant \maxfe \max\left(1,\frac{1}{\sqrt{\alpha}}\right) \frac{\pi}{2} \sqrt{1 - 2 \varphi(1)}.
\end{align*}

In conclusion, we obtained the uniform bound
\begin{align*}
	n_i(x) \leqslant \frac{4}{\alpha^{1/4}} \left(-2\varphi(1)\right)^{1/4} + 4 \maxfe \max\left(1,\frac{1}{\sqrt{\alpha}}\right) \frac{\pi}{2} \sqrt{1 - 2 \varphi(1)}.
\end{align*}

\subsection{Lower bound on $n_i$}

Let us consider that $f_{e,b}$ is positive on a segment. More precisely, we assume that there exists $0 \leqslant \domfel < \domfeu \leqslant \ve$ and $\minfe > 0$ such that $f_e([-\domfeu,-\domfel]) \geqslant \minfe$.
The value $\minfe$ is propagated along every electron characteristic crossing $(x=0,v\in[-\domfeu,-\domfel])$, so that we may use the lower estimate
\begin{align*}
	f_e(x,v) \geqslant \minfe \,\textbf{1}_{\left\{\mathcal{L}_e(0,\domfel) \leqslant \mathcal{L}_e(x,v) \leqslant \mathcal{L}_e(0,\domfeu)\right\}}.
\end{align*}
Let us parametrize the electron characteristic issued from $(0,-\domfeu)$ by $(\overline{y}(v), v)$, with $v\in[-\domfeu,0]$. Let $x\in[0,1]$, and define $\underline{w}$ and $\overline{w}$ by
\begin{align*}
	\begin{cases}
		\mathcal{L}_i(0,-\domfeu) = \mathcal{L}_i(x,-\underline{w}), \\
		\mathcal{L}_i(0,-\domfel) = \mathcal{L}_i(x,-\overline{w})
	\end{cases}
	\quad \text{that is} \quad
	\begin{cases}
		\underline{w} \coloneqq \left(\frac{\domfeu^2}{2} - 2 \varphi(x)\right)^{1/2} \\
		\overline{w} \coloneqq \left(\frac{\domfel^2}{2} - 2 \varphi(x)\right)^{1/2}
	\end{cases}
\end{align*}

\begin{figure}
	\centering
	\includegraphics[width=0.5\linewidth]{images/fpcharmaps_lowerboundni}
	\caption{Notations for the lower bound on $n_i$.}
	\mysubcaption{The coloured area corresponds to the domain $\mathcal{L}_e(0,\domfel) \leqslant \mathcal{L}_e(x,v) \leqslant \mathcal{L}_e(0,\domfeu)$, on which we know that $f_e \geqslant \minfe$. The parametrization $(\overline{y}(w),w)$ of the electron characteristic crossing $(0,-\domfeu)$ is equivalent to $(y,-h(y))$.}
	\label{fig:charmaps_lowerboundni}
\end{figure}

Then, using the (now classical) reparametrization, the ion density satisfies
\begin{align*}
	n_i(x) \geqslant \minfe \int_{w=-\overline{w}}^{-\underline{w}} \int_{y=0}^{\overline{y}(w)} \frac{1}{\left(w^2 + 2\left(\varphi(x) - \varphi(y)\right)\right)^{1/2}} dy dw.
\end{align*}
As for the integral $\IntLow$, we wish to use the velocity at $x=0$ instead of $x=x$. We define $v\in[-\domfeu,-\domfel]$ by
\begin{align*}
	\mathcal{L}_i(0,v) = \mathcal{L}_i(x,w), \quad w = - \left(v^2 - 2 \varphi(x)\right)^{1/2}, \quad dw = \frac{- v}{\left(v^2 - 2 \varphi(x)\right)^{1/2}}.
\end{align*}
This yields
\begin{align*}
	n_i(x) \geqslant \minfe \int_{v=-\domfeu}^{-\domfel} \int_{y=0}^{\overline{y}(w(v))} \frac{1}{\left(v^2 - 2\varphi(y)\right)^{1/2}} dy  \frac{-v}{\left(v^2 - 2 \varphi(x)\right)^{1/2}} dv.
\end{align*}
The upper bound $\overline{y}(w(v))$ is equal to $\varphi^{-1}\left(\frac{\mu}{2}\left(v^2 - \domfeu^2\right)\right)$.
Since $-\varphi(z) \leqslant -\varphi(1)$ for all $z$, this gives
\begin{align*}
	n_i(x) \geqslant \minfe \int_{v=-\domfeu}^{-\domfel} \overline{y}(w(v)) \frac{-v}{v^2 - 2\varphi(1)} dv = \minfe \int_{v=-\domfeu}^{-\domfel} \varphi^{-1}\left(\frac{\mu}{2}\left(v^2 - \domfeu^2\right)\right) \frac{-v}{v^2 - 2\varphi(1)} dv.
\end{align*}

Let $v_* \in ]\domfel,\domfeu[$ be the velocity such that $v_*^2 - \domfel^2 = \frac{\domfeu^2 - \domfel^2}{2}$. We split the integral on $[-\domfeu,-v_*[ \cup [-v_*,\domfel]$ and use the decreasing monotonicity of $\varphi^{-1}$ to write
\begin{align*}
	n_i(x) 
	&\geqslant 0 + \varphi^{-1}\left(\frac{\mu}{2}\left(v_*^2 - \domfeu^2\right)\right) \minfe \int_{v=-v_*}^{-\domfel} \frac{-v}{v^2 - 2\varphi(1)} dv \\
	&= \frac{\minfe}{2} \varphi^{-1}\left(\frac{\mu}{4}\left(\domfel^2 - \domfeu^2\right)\right) \log\left(\frac{v_*^2 - 2 \varphi(1)}{\domfel^2 - 2 \varphi(1)}\right) \\
	&\geqslant \frac{\minfe}{2} \varphi^{-1}\left(\frac{\mu}{4}\left(\domfel^2 - \domfeu^2\right)\right) \log\left(1 + \frac{\domfeu^2 - \domfel^2}{\domfel^2 - 2 \varphi(1)}\right).
\end{align*}

\fi

%Using Fubini's theorem, we may rewrite this as
%\begin{align*}
%	n_i(x) \geqslant \minfe  \int_{y=0}^{\overline{y}(-\domfel)} \int_{w=-\overline{w}}^{-h(y)} \frac{1}{\left(w^2 + 2\left(\varphi(x) - \varphi(y)\right)\right)^{1/2}} dw dy,
%\end{align*}
%where $y = \overline{y}(w) \coloneqq \varphi^{-1}\left(\mu \frac{w^2}{2} - \mu \frac{\domfeu^2}{2}\right)^{1/2}$ is equivalent to $w = - h(y) \coloneqq -\left(\domfeu^2 + \frac{2}{\mu} \varphi(y)\right)^{1/2}$.


%\bibliographystyle{alpha}
%\bibliography{CEMRACS.bib}

\end{document}