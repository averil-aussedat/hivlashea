\documentclass{article}

%%%%%%%%%%%%%%%%%%%%%%%%%%%%%%%%%%%%%%%%%%%%%%%%%%%%%%%%
% Packages
%%%%%%%%%%%%%%%%%%%%%%%%%%%%%%%%%%%%%%%%%%%%%%%%%%%%%%%%

%\usepackage[utf8]{inputenc} % for overleaf/PLM
\usepackage[latin1]{inputenc} % averil local
\usepackage[T1]{fontenc} % hyphenation
\usepackage{fullpage} % DO NOT USE IN BEAMER
\usepackage[british,UKenglish,USenglish,american]{babel}
%\usepackage{appendix}
\usepackage{amssymb,amsmath,amsthm,enumerate}
\usepackage{mathtools} % coloneqq
%\usepackage{easybmat}
\usepackage{enumitem}
%\usepackage{tikz}
%\usepackage{caption}
\usepackage{float} % [H]
%\usepackage{bbold}
\usepackage{xcolor}
\usepackage{stmaryrd} % ll/rr brackets
%\usepackage[notcite, notref]{showkeys}
%\usepackage[tworuled,vlined,nofillcomment]{algorithm2e}
\usepackage[ruled,vlined]{algorithm2e}
%\usepackage{cases} % numbered lines in cases (numcases and subnumcases)
\usepackage[overload]{empheq} % source : https://tex.stackexchange.com/questions/31951/separate-labels-in-cases
\usepackage{caption} % to have subfigures
\usepackage{subcaption} % to have subfigures
\usepackage{cleveref} % \cref. 

%%%%%%%%%%%%%%%%%%%%%%%%%%%%%%%%%%%%%%%%%%%%%%%%%%%%%%%%
% Format
%%%%%%%%%%%%%%%%%%%%%%%%%%%%%%%%%%%%%%%%%%%%%%%%%%%%%%%%

\title{Notes on fixed-point procedure}
\author{} 
\date{}

\SetKwRepeat{Do}{do}{while} % for algorithm2e package, add do-while

% set dashes instead of bullets for item lists
\setlist[itemize,1]{label=$-$}
\setlist[itemize,2]{label=$-$}
\setlist[itemize,3]{label=$-$}

% remove unnecessary formatting of clever references
\crefdefaultlabelformat{(#2#1#3)}
\crefname{equation}{}{}

%%%%%%%%%%%%%%%%%%%%%%%%%%%%%%%%%%%%%%%%%%%%%%%%%%%%%%%%
% Theorems
%%%%%%%%%%%%%%%%%%%%%%%%%%%%%%%%%%%%%%%%%%%%%%%%%%%%%%%%

\newtheorem{proposition}{Proposition}[section]
\newtheorem{definition}{Definition}[section]
\newtheorem{theoreme}{Theorem}[section]
\newtheorem{remarque}{Remark}[section]
\newtheorem{lemme}{Lemma}[section]
\numberwithin{equation}{section}

%%%%%%%%%%%%%%%%%%%%%%%%%%%%%%%%%%%%%%%%%%%%%%%%%%%%%%%%
% Commands
%%%%%%%%%%%%%%%%%%%%%%%%%%%%%%%%%%%%%%%%%%%%%%%%%%%%%%%%

\newcommand{\N}{\mathbb{N}}
\newcommand{\Z}{\mathbb{Z}}
\newcommand{\R}{\mathbb{R}}
\newcommand{\lp}{\left(}
\newcommand{\rp}{\right)}
\newcommand{\tran}[1]{\prescript{t}{}{#1}}
\newcommand{\vol}{\textup{Vol}}
\newcommand{\red}{\textcolor{red}}
\newcommand{\blue}{\textcolor{blue}}

\newcommand{\todo}[1]{{\color{red}\textbf{#1}}}
\newcommand{\vv}[1]{\begin{pmatrix} #1 \end{pmatrix}} % vector
\newcommand{\mysubeq}[2]{ % first argument : label, second : align content
	\begin{subequations}\label{#1}
		\begin{align}[left = {\empheqlbrace}]
			#2
		\end{align}
	\end{subequations}	
}
\newcommand{\mysubcaption}[1]{
	\vspace*{5pt}
	\begin{minipage}{0.8\linewidth}
		\begin{center}
			\footnotesize\emph{#1}
		\end{center}
	\end{minipage}
}
\newcommand{\imh}{\textwidth} % meant to be redefined locally

%\renewcommand\appendixpagename{Appendix}
%\renewcommand\appendixtocname{Appendix}
\renewcommand{\qedsymbol}{$\blacksquare$}

% local vocabulary
\newcommand{\ve}{\underline{v}_e}
\newcommand{\DomUpL}{\mathcal{D}_1}
\newcommand{\DomUpR}{\mathcal{D}_2}
\newcommand{\DomLow}{\mathcal{D}_3}

%%%%%%%%%%%%%%%%%%%%%%%%%%%%%%%%%%%%%%%%%%%%%%%%%%%%%%%%
%%%%%%%%%%%%%%%%%%%%%%%%%%%%%%%%%%%%%%%%%%%%%%%%%%%%%%%%
%%%%%%%%%%%%%%%%%%%%%%%%%%%%%%%%%%%%%%%%%%%%%%%%%%%%%%%%

\begin{document}

\maketitle

Let us define for all $0 < \alpha \leqslant \beta$ the set
\begin{align*}
	\mathcal{K}^{\alpha,\beta} \coloneqq \left\{\varphi \in \mathcal{C}^2\left([0,1], \mathbb{R}^{-}\right) \ |\ \varphi(0) = \varphi'(0) = 0, \quad - \beta \leqslant \varphi'' \leqslant - \alpha. \right\}
\end{align*}

\paragraph{Estimates on the set $\mathcal{K}$}

Since $\varphi : [0,1] \mapsto \mathbb{R}^-$ is decreasing, its inverse $\varphi^{-1} : \mathbb{R}^- \mapsto [0,1]$ is well-defined. By integration and using $\left(\varphi^{-1}\right)' = (\varphi'\circ \varphi^{-1})^{-1}$, we have
\mysubeq{eq:convex_bounds}{
	- \beta x &\leqslant \varphi'(x) \leqslant - \alpha x \label{eq:convex_bounds_phiprime} \\
	- \beta \frac{x^2}{2} &\leqslant \varphi(x) \leqslant - \alpha \frac{x^2}{2} \label{eq:convex_bounds_phi} \\
	\sqrt{-\frac{2y}{\beta}} &\leqslant \varphi^{-1}(y) \leqslant \sqrt{-\frac{2y}{\alpha}} \label{eq:convex_bounds_phiinv} \\
	\frac{-1}{\alpha\sqrt{-\frac{2y}{\beta}}} &\leqslant (\varphi^{-1})'(y) \leqslant \frac{-1}{\beta\sqrt{-\frac{2y}{\alpha}}} \label{eq:convex_bounds_phiinv} 
}

We want to obtain estimates on $n_i - n_e$. We make the following assumptions:
\begin{itemize}
\item The electron density $f_e$ satisfies the boundary condition, and is bounded by a constant $c \geqslant 0$. 
\item The potential $\varphi$ is strongly concave, i.e. there exists $\alpha>0$ such that $\varphi''(x) \leqslant - \alpha$ uniformly over $x\in[0,1]$.
\end{itemize}

Let us first focus on $n_e(x)$. The characteristics of the electron density are the level lines of
\begin{align}\label{eq:def_Le}
	\mathcal{L}_e(x,v) \coloneqq \frac{v^2}{2} - \frac{1}{\mu} \varphi(x).
\end{align}
Since $\varphi$ is strongly concave, these curves are closed. Since $f_e$ satisfies the homogeneous boundary condition, its support is embedded in $\left\{(x,v) \ |\ \frac{v^2}{2} - \frac{1}{\mu} \varphi(x) \leqslant \frac{0^2}{2} - \frac{1}{\mu} \varphi(1)\right\}$. In particular, we denote by $\ve$ the extremal speed of the support, 
\begin{align}\label{eq:def_ve}
	\ve \coloneqq \sqrt{-\frac{2}{\mu} \varphi(1)}.
\end{align}
We can roughly majorize
\begin{align*}
	n_e(x) = \int_{v\in\mathbb{R}} f_e(x,v) dv \leqslant \int_{v=-\ve}^{\ve} c dv = 2 c \ve \leqslant 2 c \sqrt{-\frac{2}{\mu} \varphi(1)}.
\end{align*}

The estimates on $n_i$ are slightly more technical. Let the ion Lyapunov function be defined as
\begin{align}\label{eq:def_Li}
	\mathcal{L}_i(x,v) \coloneqq \frac{v^2}{2} + \varphi(x).
\end{align}
Let $x\in[0,1]$ and $v \in \mathbb{R}_{-}$. We denote by $(x_b(x,v),v_b(x,v))$ the intersection of the boundary $\left\{x=0\right\}\cap\left\{v=0\right\}$ with the ion characteristic issued from $(x,v)$, equal to
\begin{align*}
	\vv{x_b(x,v) \\ v_b(x,v)} \coloneqq 
	\begin{cases}
		\vv{\varphi^{-1}\left(\frac{v^2}{2} + \varphi(x)\right) \\ 0} & \text{if } \mathcal{L}_i(x,v) \leqslant 0, \\
		\vv{0 \\ - \sqrt{\frac{v^2}{2} + \varphi(x)}} & \text{if } \mathcal{L}_i(x,v) > 0.
	\end{cases}
\end{align*}
In the following paragraph, we use $(x(t),v(t))_{t\leqslant0}$ to denote the characteristic reaching $(x_b(x,v),v_b(x,v))$ at $t=0$. 
We use the symmetry of $f_i$ to write 
\begin{align*}
	n_i(x) = 2 \int_{v\in\mathbb{R}^{-}} f_i(x_b(x,v), v_b(x,v)) dv = 2 \int_{v\in\mathbb{R}^{-}}\int_{t=-\infty}^{0} f_e(x(t),v(t)) dt dv.
\end{align*}
The lower bound $t \to -\infty$ is artificial, since the characteristic exits the support of $f_e$ in finite time. We will split the double integral in three domains:
\begin{enumerate}
\item $\DomUpL$ will be $\left\{(v,t) \in \mathbb{R}_{-}^2 \ |\ \mathcal{L}_i(x,v) \leqslant 0 \text{ and } x(t) \geqslant x \right\}$. This is the region contained between the $x-$axis, the critical characteristic and the vertical line going through $x$.
\item $\DomUpR$ is $\left\{(v,t) \in \mathbb{R}_{-}^2 \ |\ \mathcal{L}_i(x,v) \leqslant 0 \text{ and } x < x(t) \geqslant 1 \right\}$. It is exactly $\left\{\mathcal{L}_i \leqslant 0\right\} \setminus \DomUpL$.
\item $\DomLow$ is defined by $\left\{(v,t) \in \mathbb{R}_{-}^2 \ |\ \mathcal{L}_i(x,v) > 0 \text{ and } \ve \geqslant v(t) \right\}$. 
\end{enumerate}

\todo{Figure de la d�composition en domaines}


\bibliographystyle{alpha}
\bibliography{CEMRACS.bib}

\end{document}