\documentclass{article}

%%%%%%%%%%%%%%%%%%%%%%%%%%%%%%%%%%%%%%%%%%%%%%%%%%%%%%%%
% Packages
%%%%%%%%%%%%%%%%%%%%%%%%%%%%%%%%%%%%%%%%%%%%%%%%%%%%%%%%

%\usepackage[utf8]{inputenc} % for overleaf/PLM
\usepackage[latin1]{inputenc} % averil local
\usepackage[T1]{fontenc} % hyphenation
\usepackage{fullpage} % DO NOT USE IN BEAMER
\usepackage[british,UKenglish,USenglish,american]{babel}
%\usepackage{appendix}
\usepackage{amssymb,amsmath,amsthm,enumerate}
\usepackage{mathtools} % coloneqq
%\usepackage{easybmat}
\usepackage{enumitem}
%\usepackage{tikz}
%\usepackage{caption}
\usepackage{float} % [H]
%\usepackage{bbold}
\usepackage[dvipsnames]{xcolor}
\usepackage{stmaryrd} % ll/rr brackets
%\usepackage[notcite, notref]{showkeys}
%\usepackage[tworuled,vlined,nofillcomment]{algorithm2e}
\usepackage[ruled,vlined]{algorithm2e}
%\usepackage{cases} % numbered lines in cases (numcases and subnumcases)
\usepackage[overload]{empheq} % source : https://tex.stackexchange.com/questions/31951/separate-labels-in-cases
\usepackage{caption} % to have subfigures
\usepackage{subcaption} % to have subfigures
\usepackage{cleveref} % \cref. 

%%%%%%%%%%%%%%%%%%%%%%%%%%%%%%%%%%%%%%%%%%%%%%%%%%%%%%%%
% Format
%%%%%%%%%%%%%%%%%%%%%%%%%%%%%%%%%%%%%%%%%%%%%%%%%%%%%%%%

\title{Notes on fixed-point procedure}
\author{} 
\date{}

\SetKwRepeat{Do}{do}{while} % for algorithm2e package, add do-while

% set dashes instead of bullets for item lists
\setlist[itemize,1]{label=$-$}
\setlist[itemize,2]{label=$-$}
\setlist[itemize,3]{label=$-$}

% remove unnecessary formatting of clever references
\crefdefaultlabelformat{(#2#1#3)}
\crefname{equation}{}{}
\crefname{figure}{figure}{Figure}

%%%%%%%%%%%%%%%%%%%%%%%%%%%%%%%%%%%%%%%%%%%%%%%%%%%%%%%%
% Theorems
%%%%%%%%%%%%%%%%%%%%%%%%%%%%%%%%%%%%%%%%%%%%%%%%%%%%%%%%

\newtheorem{proposition}{Proposition}[section]
\newtheorem{definition}{Definition}[section]
\newtheorem{theoreme}{Theorem}[section]
\newtheorem{remark}{Remark}[section]
\newtheorem{lem}{Lemma}[section]
\numberwithin{equation}{section}

%%%%%%%%%%%%%%%%%%%%%%%%%%%%%%%%%%%%%%%%%%%%%%%%%%%%%%%%
% Commands
%%%%%%%%%%%%%%%%%%%%%%%%%%%%%%%%%%%%%%%%%%%%%%%%%%%%%%%%

\newcommand{\todo}[1]{{\color{red}\textbf{#1}}}
\newcommand{\vv}[1]{\begin{pmatrix} #1 \end{pmatrix}} % vector
\newcommand{\imh}{\textwidth} % meant to be redefined locally

\newcommand{\mysubeq}[2]{ % first argument : label, second : align content
	\begin{subequations}\label{#1}
		\begin{align}[left = {\empheqlbrace}]
			#2
		\end{align}
	\end{subequations}	
}
\newcommand{\mysubcaption}[1]{
	\vspace*{5pt}
	\begin{minipage}{0.8\linewidth}
		\begin{center}
			\footnotesize\emph{#1}
		\end{center}
	\end{minipage}
}
\newcommand{\myproof}[1]{
	\noindent \textbf{Demonstration}
	{\small	#1 \hfill \qedsymbol}
}
\newcommand{\intern}[1]{{\color{RoyalBlue} #1}} % will eventually be removed

% local vocabulary
\newcommand{\we}{{\underline{w}_e}} % transformation of \maxdome
\newcommand{\DomUpL}{{\mathcal{D}_1}} % domain over the critical char, x=0 side
\newcommand{\DomUpR}{{\mathcal{D}_2}} % domain over the critical char, x=1 side
\newcommand{\DomLow}{{\mathcal{D}_3}} % domain under the critical char + over -\maxdome
\newcommand{\IntUpL}{{\mathcal{I}_1}} % part of n_i on \DomUpL
\newcommand{\IntUpR}{{\mathcal{I}_2}} % part of n_i on \DomUpR
\newcommand{\IntLow}{{\mathcal{I}_3}} % part of n_i on \DomLow
\newcommand{\mindome}{{v_*}} % lower bound on a segment on which fe >=  \minfe
\newcommand{\maxdome}{{v^*}} % upper bound on a segment on which fe >=  \minfe
\newcommand{\minsuppe}{{\underline{v}}} % lower bound on a segment on which fe >=  \minfe
\newcommand{\maxsuppe}{{\overline{v}}} % upper bound on a segment on which fe >=  \minfe
\newcommand{\minfe}{{\underline{c}}} % lower bound on f_e on a segment [-\maxsuppe, -\minsuppe]
\newcommand{\maxfe}{{\overline{c}}} % upper bound on f_e
\newcommand{\lipfe}{{[f_{e,b}]}} % lipschitz constant of f_{e,b}
\newcommand{\lipfesq}{{[f_{e,b}]_{2}}} % lipschitz constant of f_{e,b}(\sqrt{})
\newcommand{\ee}{\varepsilon} % variable used when integrating with electric energy
\newcommand{\K}{{\mathcal{K}}} % convex set for fixed-point theorem

%%%%%%%%%%%%%%%%%%%%%%%%%%%%%%%%%%%%%%%%%%%%%%%%%%%%%%%%
%%%%%%%%%%%%%%%%%%%%%%%%%%%%%%%%%%%%%%%%%%%%%%%%%%%%%%%%
%%%%%%%%%%%%%%%%%%%%%%%%%%%%%%%%%%%%%%%%%%%%%%%%%%%%%%%%

\begin{document}

\maketitle

%Let us define for all $0 < \alpha \leqslant \beta$ the set
%\begin{align*}
%	\mathcal{K}^{\alpha,\beta} \coloneqq \left\{\varphi \in \mathcal{C}^2\left([0,1], \mathbb{R}^{-}\right) \ |\ \varphi(0) = \varphi'(0) = 0, \quad - \beta \leqslant \varphi'' \leqslant - \alpha. \right\}
%\end{align*}
%
%\paragraph{Estimates on the set $\mathcal{K}$}
%
%Since $\varphi : [0,1] \mapsto \mathbb{R}^-$ is decreasing, its inverse $\varphi^{-1} : \mathbb{R}^- \mapsto [0,1]$ is well-defined. By integration and using $\left(\varphi^{-1}\right)' = (\varphi'\circ \varphi^{-1})^{-1}$, we have
%\mysubeq{eq:convex_bounds}{
%	- \beta x &\leqslant \varphi'(x) \leqslant - \alpha x \label{eq:convex_bounds_phiprime} \\
%	- \beta \frac{x^2}{2} &\leqslant \varphi(x) \leqslant - \alpha \frac{x^2}{2} \label{eq:convex_bounds_phi} \\
%	\sqrt{-\frac{2y}{\beta}} &\leqslant \varphi^{-1}(y) \leqslant \sqrt{-\frac{2y}{\alpha}} \label{eq:convex_bounds_phiinv} \\
%	\frac{-1}{\alpha\sqrt{-\frac{2y}{\beta}}} &\leqslant (\varphi^{-1})'(y) \leqslant \frac{-1}{\beta\sqrt{-\frac{2y}{\alpha}}} \label{eq:convex_bounds_phiinv} 
%}

\tableofcontents

% TO WRITE
% What are the characteristics equations
% The choice of the domain $v\leqslant0$ and $x\in[0,1]$
% Symmetries

\section{Notations and assumptions}

We suppose that 
\begin{enumerate}
\item $f_{e,b}$ is nonnegative.
\item $f_{e,b}(v) = f_{e,b}(-v)$ for all $v\in\mathbb{R}$.
\item $f_{e,b}$ is compactly supported, i.e. there exists $v^* \geqslant 0$ such that $f_{e,b}(v) = 0$ for all $|v| \geqslant v^*$.
\item There exists $\maxfe \leqslant 0$ such that $f_{e,b}(v) \leqslant \maxfe$ for all $v \in \mathbb{R}$.
\item There exists $\minfe > 0$ and $0 \leqslant \minsuppe < \maxsuppe$ such that $f_{e,b}(v) \geqslant \minfe$ for all $v\in[-\maxsuppe,-\minsuppe]$.
\item The function $\varphi$ satisfies $\varphi(0)=\varphi'(0)=0$.
\item The function $\varphi$ is strongly concave, i.e. there exists $\alpha>0$ such that $\varphi((1-\tau) x + \tau y) \geqslant (1-\tau) \varphi(x)+\tau \varphi(y) + \frac{\alpha}{2} (1-\tau)\tau \left|x-y\right|^2$ for all $(x,y,\tau)\in[0,1]^3$.
\item There exists $\beta > \alpha$ such that $\varphi$ is $(-\beta)-$convex, i.e. $\varphi((1-\tau) x + \tau y) \leqslant (1-\tau) \varphi(x)+\tau \varphi(y) + \frac{\beta}{2} (1-\tau)\tau \left|x-y\right|^2$ for all $(x,y,\tau)\in[0,1]^3$. Notice that in this case, $\varphi(x) \geqslant - \beta \frac{x^2}{2}$, and in particular, $\varphi(1) \geqslant - \beta$. 
\end{enumerate}

\intern{For the reader: notations $\minsuppe,\maxsuppe$ indicates a domain on which $f_{e,b}$ is \emph{lower bounded}, while $\mindome, v^*$ denote a domain on which $f_{e,b}$ is \emph{upper bounded by 0} (with a global bound by $\maxfe$ anyway).}

We use the following notations: the Liapunov functions of the electrons and the ions are
\begin{align}\label{eq:def_Le}
	\mathcal{L}_e(x,v) \coloneqq \frac{v^2}{2} - \frac{1}{\mu} \varphi(x).
\end{align}
\begin{align}\label{eq:def_Li}
	\mathcal{L}_i(x,v) \coloneqq \frac{v^2}{2} + \varphi(x).
\end{align}

%We define an upper bound $\maxdome$ over the velocities in the support of $f_e$, given as
%\begin{align}\label{eq:def_ve}
%	\mathcal{L}_e(0,\maxdome) \coloneqq \frac{1}{\mu} \beta \geqslant -\frac{1}{\mu}\varphi(1) = \mathcal{L}_e(1,0), \quad\text{i.e.}\quad \maxdome \coloneqq \sqrt{\frac{2}{\mu}\beta}.
%\end{align}

For a given point $(x,v) \in [0,1]\times \mathbb{R}^{-}$, we denote by $(x_b, v_b)$ the intersection of the boundary $\{x\geqslant0,v=0\} \cup \{x=0,v\leqslant 0\}$ with the ion characteristic issued from $(x,v)$. The values are given by
\begin{align}\label{eq:def_xb_vb}
	\vv{x_b(x,v) \\ v_b(x,v)} \coloneqq \vv{\varphi^{-1}\left(\min\left(0,\frac{v^2}{2} + \varphi(x)\right)\right) \\ - \sqrt{\max\left(0,v^2 + 2\varphi(x)\right)}}
%	\begin{cases}
%		\vv{\varphi^{-1}\left(\frac{v^2}{2} + \varphi(x)\right) \\ 0} & \text{if } \mathcal{L}_i(x,v) \leqslant 0, \\
%		\vv{0 \\ - \sqrt{v^2 + 2\varphi(x)}} & \text{if } \mathcal{L}_i(x,v) > 0.
%	\end{cases}
\end{align}

The satisfaction of the non-emitting boundary condition in $(x=1,v\in\mathbb{R}_-)$ may interfere with the value on the line $(x=0,v\in\mathbb{R})$. \textbf{In the following notes, we consider that the non-emitting boundary condition has the precedence}, i.e. $f_e(x,v) = 0$ if $\mathcal{L}_e(x,v) \geqslant \mathcal{L}_e(1,0)$, regardless of the prescribed value of $f_{e,b}$ along this line.

\section{Estimates}

\subsection{Useful elementary lemmas}

\begin{lem}\label{lem:int_1surR_rectangle}
	Let $\alpha>0$, $\overline{x}\geqslant0$ and $\overline{y}\geqslant0$. Then
	\begin{align*}
		\mathcal{I} \coloneqq \int_{x=0}^{\overline{x}} \int_{y=0}^{\overline{y}} \frac{1}{\left(x^2+\alpha y^2\right)^{1/2}} dy dx \leqslant 2 \frac{\sqrt{2\overline{x} \overline{y}}}{\alpha^{1/4}}.
	\end{align*}
\end{lem}

\myproof{
	By the change of variable $z=\sqrt{\alpha} y$, with $\overline{z} \coloneqq \sqrt{\alpha} \overline{y}$, we have
	\begin{align*}
		\mathcal{I} = \frac{1}{\sqrt{\alpha}} \int_{x=0}^{\overline{x}} \int_{z=0}^{\overline{z}} \frac{1}{\left(x^2+ z^2\right)^{1/2}} dz dx.
	\end{align*}
	Notice that $x+z \leqslant \sqrt{2} \left(x^2 + y^2\right)^{1/2}$. Then, 
	\begin{align*}
		\mathcal{I} 
		\leqslant \frac{1}{\sqrt{\alpha}} \int_{x=0}^{\overline{x}} \int_{z=0}^{\overline{z}} \frac{\sqrt{2}}{x+z} dz dx 
		= \sqrt{\frac{2}{\alpha}} \int_{x=0}^{\overline{x}} \ln\left(\frac{x+\overline{z}}{x}\right) dx
		= \sqrt{\frac{2}{\alpha}} \int_{x=0}^{\overline{x}} \ln\left(1 + \frac{\overline{z}}{x}\right) dx.
	\end{align*}
	Using $\ln\left(1+a\right) \leqslant \sqrt{a}$, we get
	\begin{align*}
		\mathcal{I} 
		\leqslant \sqrt{\frac{2}{\alpha}} \int_{x=0}^{\overline{x}} \sqrt{\frac{\overline{z}}{x}} dx 
		= \sqrt{\frac{2}{\alpha}} 2 \sqrt{\overline{x} \overline{z}}
		= 2 \frac{\sqrt{2 \overline{x} \overline{y}}}{\alpha^{1/4}}.
	\end{align*}
}

\intern{
\begin{remark}[Exact value]
	Let $\overline{r} \coloneqq \sqrt{\overline{x}^2 + \overline{z}^2}$. Then 
	\begin{align*}
		\mathcal{I} = \overline{x} \ln\left(\frac{1 + \overline{z}/\overline{r}}{\overline{x}/\overline{r}}\right) + \overline{z} \ln\left(\frac{1 + \overline{x}/\overline{r}}{\overline{z}/\overline{r}}\right).
	\end{align*}
\end{remark}
}

\begin{lem}\label{lem:maj_dist_square}
	If $a\geqslant0$ and $b\geqslant0$, then $|a - b| \leqslant \sqrt{|a^2 - b^2|}$.
\end{lem}

\myproof{
	If $a \geqslant b$, then $|a - b| = \sqrt{(a-b)(a-b)} \leqslant \sqrt{(a - b)(a+b)} = \sqrt{a^2 - b^2}$, else $|a-b|=|b-a|$. 
}

\begin{lem}\label{lem:maj_fi}
	Let $\alpha>0$, $L \in \mathbb{R}$ and $0 \leqslant a \leqslant 1$. Suppose that $L + \alpha a^2 \geqslant 0$, and $a>0$ if $L=0$. Then
	\begin{align*}
		\mathcal{I} \coloneqq \int_{z=a}^1 \frac{1}{\left(L + \alpha z^2\right)^{1/2}} dz \leqslant 
%		\begin{cases}
%			\min\left(L^{-1/2},\frac{1+\left(\frac{L}{\alpha}\right)^{1/4}}{\sqrt{\alpha a}}\right) & \text{if } L > 0, \\
%			\frac{1}{\sqrt{\alpha a}} & \text{if } L = 0, \\
%			\min\left(|L|^{-1/2}, \sqrt{\frac{2}{\alpha a}}\right) & \text{if } L < 0.
%		\end{cases}
		\min\left(|L|^{-1/2}, \frac{2}{\sqrt{\alpha a}}\right).
	\end{align*}
%	In particular, we have the uniform bound $\mathcal{I} \leqslant \frac{2}{\sqrt{\alpha a}}$.
\end{lem}

\myproof{
	If $L>0$, we have
	\begin{align*}
		\mathcal{I} = \frac{1}{\sqrt{\alpha}} \int_{z=a}^1 \frac{1}{\left(1 + \left(\sqrt{\frac{\alpha}{L}} z\right)^2\right)^{1/2}} \frac{\sqrt{\alpha} dz}{\sqrt{L}}
		= \frac{1}{\alpha} \int_{w=a \sqrt{\frac{\alpha}{L}}}^{\sqrt{\frac{\alpha}{L}}} \frac{1}{\left(1+w^2\right)^{1/2}} dw 
		= \frac{1}{\sqrt{\alpha}} \left(\sinh^{-1}\left(\sqrt{\frac{\alpha}{L}}\right) - \sinh^{-1}\left(a\sqrt{\frac{\alpha}{L}}\right)\right).
	\end{align*}
	Using the positivity of $\sinh^{-1}\left(a\sqrt{\frac{\alpha}{L}}\right)$, and the coarse estimate $\sinh^{-1}(x) \leqslant x$, we get $\mathcal{I} \leqslant L^{-1/2}$. 
	Moreover, using that $\sinh^{-1}(x) = \log(x + \sqrt{x^2 + 1})$, we get
	\begin{align*}
		\mathcal{I} = \frac{1}{\sqrt{\alpha}} \log\left(\frac{\sqrt{\frac{\alpha}{L}} + \sqrt{\frac{\alpha}{L} + 1}}{a\sqrt{\frac{\alpha}{L}} + \sqrt{a^2\frac{\alpha}{L} + 1}}\right)
		= \frac{1}{\sqrt{\alpha}} \log\left(\frac{1 + \sqrt{1 + \frac{L}{\alpha}}}{a + \sqrt{a^2 + \frac{L}{\alpha}}}\right)
		\leqslant \frac{1}{\sqrt{\alpha}} \log\left(\frac{2 + \sqrt{\frac{L}{\alpha}}}{2 a}\right)
		\leqslant \frac{1 + \left(\frac{L}{\alpha}\right)^{1/4}}{\sqrt{\alpha a}}.
	\end{align*}
	\intern{I shamelessly used $\frac{1}{2} \leqslant 1$.} Then $\mathcal{I} \leqslant \min\left(L^{-1/2},\frac{1 + \left(\frac{L}{\alpha}\right)^{1/4}}{\sqrt{\alpha a}}\right)$ on $L>0$. But whenever $L \geqslant \alpha$, the min is attained in its first argument: indeed, $L^{-1/2} \leqslant \frac{1}{\sqrt{\alpha a}} \leqslant \frac{1+\left(\frac{L}{\alpha}\right)^{1/4}}{\sqrt{\alpha a}}$. Then we obtained $\mathcal{I} \leqslant \min(L^{-1/2},\frac{2}{\sqrt{\alpha a}})$ on $L>0$.
	If $L=0$, we have
	\begin{align*}
		\int_{z=a}^1 \frac{1}{\sqrt{\alpha} z} dz = \frac{\log(1/a)}{\sqrt{\alpha}} \frac{2}{\sqrt{\alpha a}}.
	\end{align*}
	Finally, if $L<0$, the integral becomes
	\begin{align*}
		\mathcal{I} = \frac{1}{\sqrt{\alpha}} \int_{a}^1 \frac{1}{\left(\left(\sqrt{\frac{\alpha}{|L|}} z\right)^2-1\right)^{1/2}} \frac{\sqrt{\alpha} dz}{\sqrt{|L|}}
		= \frac{1}{\alpha} \int_{a \sqrt{\frac{\alpha}{|L|}}}^{\sqrt{\frac{\alpha}{|L|}}} \frac{1}{\left(w^2-1\right)^{1/2}} dw 
		= \frac{\cosh^{-1}\left(\sqrt{\frac{\alpha}{|L|}}\right) - \cosh^{-1}\left(a\sqrt{\frac{\alpha}{|L|}}\right)}{\sqrt{\alpha}}.
	\end{align*}
	The expression is well-defined, since $L + \alpha a^2 \geqslant 0$ implies $a \sqrt{\frac{\alpha}{-L}} \geqslant 1$ when $-L>0$.
	Since $\cosh^{-1}(x) \leqslant x$, we obtain the coarse estimate $\mathcal{I} \leqslant |L|^{-1/2}$.
	Using that $\cosh^{-1}(x)=\log\left(x+\sqrt{x^2-1}\right)$, we get
	\begin{align*}
		\mathcal{I} 
		= \frac{1}{\sqrt{\alpha}} \log\left(\frac{1 + \sqrt{1 - \frac{|L|}{\alpha}}}{a + \sqrt{a^2 - \frac{|L|}{\alpha}}}\right) 
		\leqslant \frac{1}{\sqrt{\alpha}} \log\left(\frac{2}{a}\right) 
		\leqslant \sqrt{\frac{2}{\alpha a}}
		\leqslant \frac{2}{\sqrt{\alpha a}}.
	\end{align*}
	
%	To obtain the uniform bound, we just have to show that for all $L > 0$,
%	\begin{align*}
%		L^{-1/2} \textbf{1}_{\{L > c\}} + \frac{1+\left(\frac{L}{\alpha}\right)^{1/4}}{\sqrt{\alpha a}} \textbf{1}_{\{L \leqslant c\}}
%		\coloneqq \min\left(L^{-1/2},\frac{1+\left(\frac{L}{\alpha}\right)^{1/4}}{\sqrt{\alpha a}}\right) 
%		\leqslant \frac{2}{\sqrt{\alpha a}}.
%	\end{align*}
%	Notice that $L \geqslant \alpha$ implies $L^{-1/2} \leqslant \frac{1}{\sqrt{\alpha a}} \leqslant \frac{1+\left(\frac{L}{\alpha}\right)^{1/4}}{\sqrt{\alpha a}}$, so that $c \leqslant \alpha$, and the restriction $\frac{1+\left(\frac{L}{\alpha}\right)^{1/4}}{\sqrt{\alpha a}} \textbf{1}_{\{L \leqslant c\}}$ is bounded by $\frac{2}{\sqrt{\alpha a}}$. On the other hand, the inequality $L^{-1/2} \leqslant \frac{2}{\sqrt{\alpha a}}$ is satisfied as soon as $\frac{\alpha a}{4} \leqslant L$, and the critical case $L=\frac{\alpha a}{4}$ yields
%	\begin{align*}
%		\frac{1+\left(\frac{\alpha a}{4 \alpha}\right)^{1/4}}{\sqrt{\alpha a}} < \frac{2}{\sqrt{\alpha a}} = L^{-1/2}.
%	\end{align*}
%	Then $c \geqslant \frac{\alpha a}{4}$, and the bound is uniform in $L$.
}

\begin{lem}[\intern{MAGIC }Change of variable]\label{lem:magic_change_variable}
	Let $(x(\tau),v(\tau))$ be an ion characteristic issued from $(x,w) \in [0,1] \times \mathbb{R}_-$, and define $\ee \coloneqq \mathcal{L}_e(x(\tau),v(\tau))$ and $z\coloneqq x(\tau)$. Then
	\begin{align*}
		d\tau \wedge dw = \frac{dz \wedge d\ee}{2\left(\ee + \frac{1}{\mu} \varphi(z)\right)^{1/2} \left(\ee - \varphi(x) + \left(1+\frac{1}{\mu}\right) \varphi(z)\right)^{1/2}}.
	\end{align*}
\end{lem}

\myproof{
	We have $\ee = \frac{(v(\tau))^2}{2} - \frac{1}{\mu} \varphi(z)$, so that
	\begin{align*}
		d\ee = v(\tau) \dot{v}(\tau) d\tau - \frac{1}{\mu} \varphi'(z) dz = - v(\tau) \varphi'(z) d\tau - \frac{1}{\mu} \varphi'(z) dz
		\quad \implies \quad 
		d\tau = -\frac{1}{v(\tau) \varphi'(z)} d\ee - \frac{1}{\mu v(\tau)} dz.
	\end{align*}
	Moreover, $\mathcal{L}_i(x(\tau),v(\tau)) = \frac{(v(\tau))^2}{2} + \varphi(z)$, so $\left(1+\frac{1}{\mu}\right) \varphi(z) = \mathcal{L}_i(x(\tau),v(\tau)) - \mathcal{L}_e(x(\tau),v(\tau)) = \mathcal{L}_i(x,w) - \ee$ and
	\begin{align*}
		w = -\sqrt{2}\left(\ee - \varphi(x) + \left(1+\frac{1}{\mu}\right) \varphi(z)\right)^{1/2}
		\quad \implies \quad
%		dw = -\frac{d\ee + \left(1+\frac{1}{\mu}\right) \varphi'(z) dz}{\sqrt{2}\left(\ee - \varphi(x) + \left(1+\frac{1}{\mu}\right) \varphi(z)\right)^{1/2}}
		= \frac{d\ee + \left(1+\frac{1}{\mu}\right) \varphi'(z) dz}{w}.
	\end{align*}
	Then, using $v(\tau) = - \sqrt{2}\left(\ee + \frac{1}{\mu} \varphi(z)\right)^{1/2}$, we get
	\begin{align*}
		d\tau \wedge dw
		= -\frac{1+\frac{1}{\mu}}{w v(\tau)} d\ee \wedge dz - \frac{\frac{1}{\mu}}{v(\tau) w} dz \wedge d\ee
		= \frac{dz \wedge d\ee}{v(\tau) (-w)} 
		= \frac{dz \wedge d\ee}{2 \left(\ee + \frac{1}{\mu} \varphi(z)\right)^{1/2} \left(\ee - \varphi(x) + \left(1+\frac{1}{\mu}\right) \varphi(z)\right)^{1/2}}.
	\end{align*}
}

\begin{remark}\label{rem:magic_change_welldef}
	Notice that $\mathcal{L}_i(x(\tau),v(\tau)) = \ee + \left(1+\frac{1}{\mu}\right) \varphi(z)$. Moreover, we have $-\frac{1}{\mu} \varphi(z) = \mathcal{L}_e(z,0)$, and $\varphi(x) = \mathcal{L}_i(z,0)$. Then the change of variable rewrites
	\begin{align*}
		d\tau \wedge dw = \frac{dz \wedge d\ee}{2\left(\mathcal{L}_e(x(\tau),v(\tau)) - \mathcal{L}_e(z,0)\right)^{1/2} \left(\mathcal{L}_i(x(\tau),v(\tau)) - \mathcal{L}_i(x,0)\right)^{1/2}},
	\end{align*}
	and is well-defined for $(\tau,w)$ such that both square roots are real.
\end{remark}

\subsection{Boundedness estimates}

\subsubsection{Electronic density $n_e$}

\begin{lem}\label{lem:ne_continuous_space}
	The electron density $n_e$ is bounded independently of $\varphi$. More precisely, 
	\begin{align*}
		n_e(x) \leqslant 2 \maxfe \maxdome \quad\quad\forall x \in[0,1].
	\end{align*}
\end{lem}

\myproof{
	Let $v\in\mathbb{R}$ such that $|v| > \mindome$, with $f_{e,b}\left(\mathbb{R}\setminus[-\maxdome,\maxdome]\right) = 0$. Then 
	\begin{align*}
		\mathcal{L}_e(x,v) = \frac{v^2}{2} - \frac{1}{\mu} \varphi(x) \geqslant \frac{v^2}{2} = \mathcal{L}_e(v,0)
		\quad\implies\quad
		f_e(x,v) = 0.
	\end{align*}
	Then we may write for all $x\in[0,1]$ that
	\begin{align*}
		n_e(x) = \int_{v=-\maxdome}^{\maxdome} f_e(x,v) dv \leqslant 2 \maxfe \maxdome.
	\end{align*}
}

\subsubsection{Ion density $n_i$}

The estimates will rely on two particular cases, the we treat independently as lemmas. For a given $x$, we define $g_x : [0,x] \mapsto \mathbb{R}^{+}$ by
\begin{align*}
	\mathcal{L}_i(x,-g_x(y)) = \mathcal{L}_i(y,0), \quad \text{i.e.} \quad g_x(y) = \left(2\left(\varphi(y) - \varphi(x)\right)\right)^{1/2}.
\end{align*}

\intern{For the reader: draw $y$ to the left of $x$. Then $-g_x(y)$ is the intersection of the ion characteristic ending on $(y,0)$ with the vertical line going through $x$.}

\begin{lem}\label{lem:upperbound_ni_endchar}
	Let $0\leqslant y < x \leqslant 1$. We have
	\begin{align*}
		\mathcal{I} \coloneqq \int_{v=-g_x(y)}^{0} \int_{z = x_b(x,v)}^{x} \frac{1}{\left(v^2-g_x^2(z)\right)^{1/2}} dz\,dv \leqslant 2 \left(\frac{\beta}{\alpha}\right)^{1/4} (x-y). 
	\end{align*}
\end{lem}

\myproof{
	Let us first use Fubini's theorem to switch the order of integration. The lower bound $x_b(x,v) \geqslant z$ becomes an upper bound $v \leqslant -g_x(z)$, and we have
	\begin{align*}
		\mathcal{I} 
		= \int_{z=y}^{x} \int_{v = -g_x(y)}^{-g_x(z)} \frac{1}{\left(v^2-g_x^2(z)\right)^{1/2}} dv\,dz 
		= \int_{z=y}^{x} \int_{v = g_x(z)}^{g_x(y)} \frac{1}{\left(v^2-g_x^2(z)\right)^{1/2}} dv\,dz.
	\end{align*} 
	With the change of variable $w = v - g_x(z)$, and using $\frac{d}{dw} \left[\sinh^{-1}\left(\sqrt{\frac{w}{a}}\right)\right] = \left(w^2 + 2aw\right)^{-1/2}$, we get
	\begin{align*}
		\mathcal{I} 
		= \int_{z=y}^{x} \int_{w = 0}^{g_x(y)-g_x(z)} \frac{1}{\left(w^2 + 2 w g_x(z)\right)^{1/2}} dw\,dz
		= \int_{z=y}^{x} \sinh^{-1}\left(\sqrt{\frac{g_x(y) -g_x(z)}{g_x(z)}}\right) dz.
	\end{align*}
	\intern{
	Using the elementary estimate $\sinh^{-1}(a) \leqslant p\,a^{1/p}$ for all $p\in\mathbb{N}^*$, and the coarse estimate $\sqrt{\frac{a-b}{b}} \leqslant \sqrt{\frac{a}{b}}$, we get
	\begin{align*}
		\mathcal{I} 
		\leqslant \int_{z=y}^x p \left(\frac{g_x(y)}{g_x(z)}\right)^{1/2p} dz 
		= \int_{z=y}^x p \left(\frac{\varphi(y) - \varphi(x)}{\varphi(z) - \varphi(x)}\right)^{1/4p} dz
		 \quad\forall\,p\in\mathbb{N}^*.
	\end{align*}
	The assumption of strong concavity yields $\varphi(z) - \varphi(x) \geqslant - \varphi'(z) (x - z) + \frac{\alpha}{2} |x - z|^2 \geqslant \frac{\alpha}{2} (x - z)^2$, so that 
	\begin{align*}
		\mathcal{I} 
		\leqslant \left(\frac{2}{\alpha}\right)^{1/4p} (\varphi(y) - \varphi(x))^{1/4p} \int_{z=y}^x p \frac{1}{(x-z)^{1/2p}} dz
		= \left(\frac{2}{\alpha}\right)^{1/4p} (\varphi(y) - \varphi(x))^{1/4p} \frac{p^2}{p-1/2)} (x-y)^{1-1/2p}.
	\end{align*}
	By the strong $(-\beta)-$convexity, we may write $\varphi(y) - \varphi(x) \leqslant \beta \frac{(x-y)^2}{2}$, so that
	\begin{align*}
		\mathcal{I} \leqslant \left(\frac{\beta}{\alpha}\right)^{1/4p} \frac{p^2}{p-1/2} (x-y) \quad \forall p \in \mathbb{N}^*.
	\end{align*}
	}
	Using the coarse estimates $\sinh^{-1}(a) \leqslant a$ and $\sqrt{\frac{a-b}{b}} \leqslant \sqrt{\frac{a}{b}}$, we get
	\begin{align*}
		\mathcal{I} \leqslant \int_{z=y}^{x} \sqrt{\frac{g_x(y)}{g_x(z)}} dz = \int_{z=y}^{x} \left(\frac{\varphi(y) - \varphi(x)}{\varphi(z) - \varphi(x)}\right)^{1/4} dz.
	\end{align*}
	The assumption of strong concavity yields $\varphi(z) - \varphi(x) \geqslant - \varphi'(z) (x - z) + \frac{\alpha}{2} |x - z|^2 \geqslant \frac{\alpha}{2} (x - z)^2$, so that 
	\begin{align*}
		\mathcal{I} 
		\leqslant \left(\frac{2}{\alpha}\right)^{1/4} \left(\varphi(y) - \varphi(x)\right)^{1/4} \int_{z=y}^{x} \frac{1}{\left(x - z\right)^{1/2}} dz 
		= 2 \left(\frac{2}{\alpha}\right)^{1/4} \left(\varphi(y) - \varphi(x)\right)^{1/4} \sqrt{x-y}.
	\end{align*}
	By the strong $(-\beta)-$convexity, we may write $\varphi(y) - \varphi(x) \leqslant \beta \frac{(x-y)^2}{2}$, so that
	\begin{align*}
		\mathcal{I} \leqslant 2 \left(\frac{\beta}{\alpha}\right)^{1/4} (x-y).
	\end{align*}
}

\begin{lem}\label{lem:upperbound_ni_beginchar}
	Let $0 \leqslant y \leqslant x \leqslant 1$, and $-v_0 < -g_x(y)$. We have
	\begin{align*}
		\mathcal{I} 
		\coloneqq \int_{v=-v_0}^{-g_x(y)} \int_{z=y}^{1} \frac{1}{\left(v^2 + 2 \left(\varphi(x) - \varphi(z)\right)\right)^{1/2}} dz \, dv 
%		\leqslant \frac{2\sqrt{2(1-y)}}{\alpha^{1/4}} \left(v_0^2 + 2 \left(\varphi(x) - \varphi(y)\right)\right)^{1/4}.
		\leqslant \frac{2\sqrt{2}}{\alpha^{1/4}} \left(v_0^2 + \beta \left(x - y\right)^2\right)^{1/4}.
	\end{align*}
\end{lem}

\myproof{
	We first shift the $v-$integration from the vertical line $z=x$ to $z=y$. Let $w = w(v)$ be such that
	\begin{align*}
		\mathcal{L}_i(y,w(v)) = \mathcal{L}_i(x,v), \quad \text{i.e.} \quad w(v) = -\left(v^2 + 2 \left(\varphi(x) - \varphi(y)\right)\right), \text{ and } dv = \frac{-w}{\left(w^2 + 2 \left(\varphi(y) - \varphi(x)\right)\right)^{1/2}}  dw.
	\end{align*}
	Then, defining $w_0 \coloneqq \left(v_0^2 + 2 \left(\varphi(x) - \varphi(y)\right)\right)^{1/2}$, and noticing that $w(-g_x(y)) = 0$, we get
	\begin{align*}
		\mathcal{I} = \int_{w=-w_0}^{0} \int_{z=y}^{1} \frac{1}{\left(w^2 + 2 \left(\varphi(y) - \varphi(z)\right)\right)^{1/2}} dz \frac{-w}{\left(w^2 + 2 \left(\varphi(y) - \varphi(x)\right)\right)^{1/2}} dw.
	\end{align*}
	Since $y \leqslant x$, we have $\varphi(y) \geqslant \varphi(x)$, and $\frac{-w}{\left(w^2 + 2 \left(\varphi(y) - \varphi(x)\right)\right)^{1/2}} \leqslant \frac{-w}{|w|} = 1$. By the strong concavity assumption, we have $\varphi(y) - \varphi(z) \geqslant - \varphi'(y) (z - y) + \frac{\alpha}{2} |z - y|^2 \geqslant \frac{\alpha}{2} (z - y)^2$, so that
	\begin{align*}
		\mathcal{I} \leqslant \int_{w=-w_0}^{0} \int_{z=y}^{1} \frac{1}{\left(w^2 + \alpha (z- y)^2\right)^{1/2}} dz dw = \int_{w=0}^{w_0} \int_{z=0}^{1-y} \frac{1}{\left(w^2 + \alpha z^2\right)^{1/2}} dz dw.
	\end{align*}
	Using \cref{lem:int_1surR_rectangle}, and the $(-\beta)-$convexity, we conclude that 
	\begin{align*}
		\mathcal{I} 
		\leqslant 2 \frac{\sqrt{2 w_0 (1-y)}}{\alpha^{1/4}} 
		= \frac{2\sqrt{2(1-y)}}{\alpha^{1/4}} \left(v_0^2 + 2 \left(\varphi(x) - \varphi(y)\right)\right)^{1/4}
		\leqslant \frac{2\sqrt{2}}{\alpha^{1/4}} \left(v_0^2 + \beta \left(x - y\right)^2\right)^{1/4}.
	\end{align*}
}

\begin{proposition}
	The density $n_i$ is bounded. More precisely,
	\begin{align*}
		n_i(x) 
%		\leqslant 8 \nu \left(\frac{\beta}{\alpha}\right)^{1/4} \left(\frac{1}{\alpha^{1/4}} + \beta^{1/4} + \frac{1}{\mu^{1/4}}\right) 
		\leqslant 2\nu \left(2 \left(\frac{\beta}{\alpha}\right)^{1/4} + 2\sqrt{2} \left(\frac{\beta}{\alpha}\right)^{1/4} + \frac{2\sqrt{2}}{\alpha^{1/4}} \sqrt{\maxdome}\right)
		\quad\quad\forall x \in[0,1].
	\end{align*}
\end{proposition}

\myproof{
	We use the symmetry of $f_i$ to write
	\begin{align*}
		n_i(x) = \int_{v=-\infty}^{\infty} f_i(x,v) dv = 2 \int_{v=-\infty}^{0} f_i(x_b(x,v), v_b(x,v)) dv = 2 \nu \int_{v=-\infty}^{0} \int_{t=-\infty}^{0} f_e(x(t),v(t)) dt dv,
	\end{align*}
	where $(x(t),v(t))_{t\leqslant0}$ is the ion characteristic reaching $(x_b(x,v),v_b(x,v))$ at $t=0$. Notice that the lower bounds are artificial, since the characteristic enters the support of $f_e$ in finite time: we may use $v \geqslant -\maxdome$, and consider only times $t$ for which $x(t) \in [0,1]$. 
	
	We first reparametrize $(x(t),v(t))$ using the space variable. Define $z = x(t) \in[x_b,1]$, and observe that
	\begin{align*}
		dz = \dot{x}(t) dt = v(t) dt, \quad \text{with}\quad \mathcal{L}_i(z,v(t)) = \mathcal{L}_i(x,v) \quad\iff\quad v(t) = -\left(v^2 + 2\left(\varphi(x) - \varphi(z)\right)\right)^{1/2}.
	\end{align*}
	Then, the density rewrites
	\begin{align*}
		n_i(x) = 2 \nu \int_{v=-\maxdome}^{0} \int_{z=x_b(x,v)}^{1} \frac{f_e\left(z, -\left(v^2 + 2\left(\varphi(x) - \varphi(z)\right)\right)^{1/2}\right)}{\left(v^2 + 2\left(\varphi(x) - \varphi(z)\right)\right)^{1/2}} \, dz dv.
	\end{align*}
	Let us show that $n_i$ is bounded. We use the coarse estimate $f_e \leqslant \maxfe$, and decompose the integral in three:
	\begin{align*}
		n_i(x) \leqslant 2 \nu \maxfe \left[
		\underbrace{\int_{v=-g_x(0)}^{0} \int_{z=x_b(x,v)}^{x}}_{\IntUpL} + 
		\underbrace{\int_{v=-g_x(0)}^{0} \int_{z=x}^{1}}_{\IntUpR} + 
		\underbrace{\int_{v=-\maxdome}^{-g_x(0)} \int_{z=x_b(x,v)}^{1}}_{\IntLow} \right] 
		\frac{1}{\left(v^2 + 2\left(\varphi(x) - \varphi(z)\right)\right)^{1/2}} \, dz dv.
	\end{align*}
	The corresponding domains are represented \cref{fig:charmaps_domainmap}. 
	\begin{figure}
		\centering
		\includegraphics[width=0.5\linewidth]{images/fpcharmaps_domainmap}
		\caption{Decomposition of the integral defining $n_i$.}
		\mysubcaption{The phase space is divided by the critical characteristic (in solid black). Whenever $v\leqslant -g_x(0)$, the characteristics (dotted green lines) are reaching the boundary with $x_b(x,v)=0$. }
		\label{fig:charmaps_domainmap}
	\end{figure}
	
	Notice that whenever $z \leqslant x$, we have $0 \geqslant 2\left(\varphi(x) - \varphi(z)\right) = - \left(2\left(\varphi(z) - \varphi(x)\right)\right)^{2/2} = - g_x^2(z)$. Then, the integral $\IntUpL$ may be bounded using \cref{lem:upperbound_ni_endchar} with $y=0$:
	\begin{align*}
		\IntUpL = \int_{v=-g_x(0)}^{0} \int_{z=x_b(x,v)}^{x} \frac{1}{\left(v^2 - g_x^2(z)\right)^{1/2}} \, dz dv 
		\leqslant 2 \left(\frac{\beta}{\alpha}\right)^{1/4} x
		\leqslant 2 \left(\frac{\beta}{\alpha}\right)^{1/4}.
	\end{align*}
	
	We use \cref{lem:upperbound_ni_beginchar} to bound $\IntUpR$ and $\IntLow$. In the first case, we take $y=x$ and $v_0 = g_x(0)$, and notice that $-g_x(x)=0$. In the second case, we take $v_0 = \maxdome$ and $y=0$, and notice that on $v \leqslant -g_x(0)$, we have $x_b(x,v) = 0$ \intern{(the velocity is low enough so that the characteristic ends on $x_b=0$)}. Finally, we use that $g_x^2(0) = -2\varphi(1) \leqslant \beta$ to write
	\begin{align*}
%		\IntUpR \leqslant \frac{2\sqrt{2(1-x)}}{\alpha^{1/4}} \left(g_x^2(0)\right)^{1/4} \leqslant \frac{4}{\alpha^{1/4}} \left(-\varphi(1)\right)^{1/2} , \quad\text{and}\quad
%		\IntLow \leqslant \frac{2\sqrt{2}}{\alpha^{1/4}} \left(\maxdome^2 + 2 \varphi(x)\right)^{1/4} \leqslant \frac{4}{\alpha^{1/4}} \left(\frac{\beta}{\mu}\right)^{1/4}.
		\IntUpR \leqslant \frac{2\sqrt{2}}{\alpha^{1/4}} \left(g_x^2(0)\right)^{1/4} \leqslant 2\sqrt{2} \left(\frac{\beta}{\alpha}\right)^{1/4}, \quad\text{and}\quad
		\IntLow \leqslant \frac{2\sqrt{2}}{\alpha^{1/4}} \left((\maxdome)^2 + 2 \varphi(x)\right)^{1/4} \leqslant \frac{2\sqrt{2}}{\alpha^{1/4}} \sqrt{\maxdome}.
	\end{align*}
}

\begin{lem}
	The density $n_i$ is uniformly bounded away from 0. More precisely, 
	\begin{align*}
		n_i(x) \geqslant \frac{\sqrt{2} \nu \minfe (\maxsuppe^2 - \minsuppe^2)}{\maxsuppe \left(\frac{\maxsuppe^2}{2} + \beta\right)^{1/2}} \frac{\minsuppe}{\sqrt{\beta\left(1+\frac{1}{\mu}\right)}} \quad\quad\forall x\in[0,1].
	\end{align*}
\end{lem}

\myproof{
	By assumption, there exists a constant $\minfe>0$ such that $f_{e,b} (v) \geqslant \minfe \textbf{1}_{\left\{\minsuppe \leqslant |v| \leqslant \maxsuppe\right\}}$, which implies
	\begin{align*}
		f_e(x,v) \geqslant \minfe \textbf{1}_{\left\{\mathcal{L}_e(0,\minsuppe) \leqslant \mathcal{L}_e(x,v) \leqslant \mathcal{L}_e(0,\maxsuppe)\right\}}.
	\end{align*}
	Notice that we may always take $\minsuppe > 0$ as long as $\maxsuppe>0$.
	Let us denote $\underline{\ee} \coloneqq \mathcal{L}_e(0,\minsuppe)$ and $\overline{\ee} \coloneqq \mathcal{L}_e(0,\maxsuppe)$. For any $x\in[0,1]$, we have
	\begin{align*}
		n_i(x) = 2 \nu \int_{w\in\mathbb{R}_-} \int_{\tau=-\infty}^{0} f_e(x(\tau),v(\tau)) d\tau dw, \quad \text{with} \quad \mathcal{L}_i(x(\tau),v(\tau)) = \mathcal{L}_i(x,w).
	\end{align*}
	Notice that for $w \in [-g_x(-\maxsuppe),-g_x(0)]$, the characteristic issued from $(x,w)$ enters the domain $\left\{\underline{\ee} \leqslant \mathcal{L}_e \leqslant \overline{\ee}\right\}$. We will restrict the domain of integration to 
	\begin{align*}
		\mathcal{D} \coloneqq \left\{(w,\tau) \in \mathbb{R}_-^2 \ |\ \underline{\ee} \leqslant \mathcal{L}_e(x(\tau),v(\tau)) \leqslant \overline{\ee} \text{ and } 0 \leqslant x(\tau) \leqslant a\right\},
	\end{align*}
	where $(a,v_a)$ is the unique point such that $\mathcal{L}_e(a,v_a) = \mathcal{L}_e(0,\minsuppe)$ and $\mathcal{L}_i(a,v_a) = 0$ (note that $a$ depends on $\beta$). This (somehow coarse) estimate will give us trivial bounds when applying the change of variable of \cref{lem:magic_change_variable}, namely $z=x(\tau)$ and $\ee = \mathcal{L}_e(x(\tau),v(\tau))$. Denoting by $[\underline{\tau}, \overline{\tau}] \subset ]-\infty,0]$ the time interval where $(x(\tau),v(\tau)) \in \mathcal{D}$, we have
	\begin{align*}
		n_i(x) 
		\geqslant \int_{w=-g_x(-\maxsuppe)}^{-g_x(0)} \int_{\tau = \underline{\tau}}^{\overline{\tau}} 2 \nu \minfe \, d\tau dw
		= \int_{\ee = \underline{\ee}}^{\overline{\ee}} \int_{z=0}^{a} \frac{2 \nu \minfe}{2\left(\ee + \frac{1}{\mu} \varphi(z)\right)^{1/2} \left(\ee - \varphi(x) + \left(1+\frac{1}{\mu}\right) \varphi(z)\right)^{1/2}} \, dz d\ee.
	\end{align*}
	Using \cref{rem:magic_change_welldef}, we are assured that the square roots are well-defined, since 
	\begin{align*}
		\ee \geqslant \underline{\ee} = \mathcal{L}_e(0,\underline{v}) = \mathcal{L}_e(a,v_a) \geqslant \mathcal{L}_e(a,0) \geqslant \mathcal{L}_e(z,0) = -\frac{1}{\mu} \varphi(z)
		\quad\text{and}\quad
		\mathcal{L}_i(x(\tau),v(\tau)) = \mathcal{L}_i(x,w) \geqslant \mathcal{L}_i(x,0).
	\end{align*}
	Using that $\ee + \frac{1}{\mu}\varphi(z) \leqslant \overline{\ee} + 0$ and $\ee - \varphi(x) + \left(1+\frac{1}{\mu} \varphi(z)\right) \leqslant \overline{\ee} + \beta + 0$, we obtain
	\begin{align*}
		n_i(x) 
		\geqslant 2 \nu \int_{\ee = \underline{\ee}}^{\overline{\ee}} \int_{z=0}^{a} \frac{\minfe}{2\overline{\ee}^{1/2}\left(\overline{\ee} + \beta\right)^{1/2}} \, dz d\ee
%		= \frac{a \minfe}{\overline{\ee}^{1/2}} \left[\sqrt{\ee + \beta}\right]_{\underline{\ee}}^{\overline{\ee}}.
		= \frac{\nu a \minfe (\overline{\ee} - \underline{\ee})}{\overline{\ee}^{1/2} \left(\overline{\ee} + \beta\right)^{1/2}}.
	\end{align*}
	Let us make explicit the dependance over $\beta$ by estimating $a$ using the $(-\beta)-$convexity assumption: we have
	\begin{align*}
		 \varphi(x) \geqslant - \beta \frac{x^2}{2}
		 \quad \implies \quad
		 \varphi^{-1}(y) \geqslant \sqrt{\frac{- 2 y}{\beta}}, 
		 \quad\text{so that}\quad
 		 a = \varphi^{-1}\left(-\frac{\minsuppe^2}{2\left(1+\frac{1}{\mu}\right)}\right)
 		 \geqslant \frac{\minsuppe}{\sqrt{\beta\left(1+\frac{1}{\mu}\right)}}.
	\end{align*}
	and the uniform lower bound behaves like $\beta^{-1}$ when $\beta\to\infty$.
}

\begin{figure}
	\centering
	\includegraphics[width=0.5\linewidth]{images/fpcharmaps_lowerboundni}
	\caption{Notations for the lower bound on $n_i$.}
	\mysubcaption{The coloured area corresponds to the domain $\underline{\ee} \leqslant \mathcal{L}_e \leqslant \overline{\ee}$, on which we know that $f_e \geqslant \minfe$. The hatched area represents the domain $\mathcal{D}$. The solid black line is the critical ion characteristic.}
	\label{fig:charmaps_lowerboundni}
\end{figure}

\begin{lem}
	\todo{We probably can do better, but maybe not win any exponent.}
	Suppose that there exists $\mindome > 0$ such that $f_{e,b}(v) = 0$ for all $|v| \leqslant \mindome$. Then the density $f_i$ satisfies the uniform bound
	\begin{align*}
		f_i(x,v) \leqslant 2^{9/4} \maxfe \sqrt{\frac{\frac{\beta}{\alpha}(1+\frac{1}{\mu})+1}{\mindome}} \quad \forall (x,v) \in [0,1] \times \mathbb{R}.
	\end{align*}
\end{lem}

\myproof{
	Let $(x,v) \in [0,1] \times \mathbb{R}_{-}$, and denote by $(x(t),v(t))_{t\leqslant 0}$ the ion characteristic going through $(x,v) = (x(-T),v(-T))$, with the convention $(x_b(x,v),v_b(x,v)) = (x(0),v(0))$. Owing to the positivity of $f_e$, 
	\begin{align*}
		f_i(x(0),v(0)) = \int_{t=-T}^{0} f_e(x(t),v(t)) dt + f_i(x,v) \geqslant f_i(x,v).
	\end{align*}
	On the other hand, we use $f_i(x,v) + f_i(x,-v) = 2 f_i(x(0),v(0))$ to write $f_i(x,-v) \leqslant 2 f_i(x(0),v(0))$. This shows that it is enough to bound $f_i$ on the boundary $\mathcal{B} \coloneqq \{x\geqslant0,v=0\} \cup \{x=0,v\leqslant 0\}$ to obtain an uniform bound.
	
	Let then $(x,v) \in \mathcal{B}$. By hypothesis, $f_e(x(t),v(t))$ vanishes whenever $\mathcal{L}_e(x(t),v(t)) \notin [\mathcal{L}_e(0,\mindome), \mathcal{L}_e(0,\maxdome)]$, and is bounded by $\maxfe$ otherwise. Then, we may use the reparametrization by space 
	\begin{align*}
		f_i(x,v) \leqslant \int_{z=a(x,v)}^{1} \frac{\maxfe}{\left(v^2 + 2 \varphi(x) - 2 \varphi(z)\right)^{1/2}} dz,
		\quad \text{with} \quad
		a(x,v) \coloneqq 
		\begin{cases}
			\varphi^{-1}\left(\frac{\frac{v^2}{2} + \varphi(x) - \frac{\mindome^2}{2}}{1+1/\mu}\right) & \text{if } \mathcal{L}_e(x,v) \leqslant \mathcal{L}_e(0,\mindome) \\
			x & \text{otherwise.}
		\end{cases}
	\end{align*}
	The function $a$ gives the smallest spatial coordinate of the characteristic $(x(t),v(t))_{t\leqslant0}$ such that $f_e > 0$. Using that $\varphi(z) \leqslant - \alpha\frac{z^2}{2}$, we have
	\begin{align*}
		f_i(x,v) \leqslant \maxfe \int_{z=a(x,v)}^{1} \frac{1}{\left(v^2 + 2 \varphi(x) - \alpha z^2\right)^{1/2}} dz \leqslant \maxfe \min\left(\frac{1}{\left|v^2 + 2 \varphi(x)\right|^{1/2}}, \frac{2}{\sqrt{\alpha a(x,v)}}\right)
	\end{align*}
	where we used \cref{lem:maj_fi} with $L\coloneqq v^2 + 2 \varphi(x)$. 
	
	We rely on the concavity estimate
	\begin{align*}
		\varphi(x) \geqslant (1 - x) \varphi(0) + x \varphi(1) + \frac{\alpha}{2} x(1-x) \geqslant - x \beta 
		\quad \implies \quad
		-\frac{y}{\beta} \leqslant \varphi^{-1}(y)
	\end{align*}
	to write that for $(x,v)$ satisfying $\mathcal{L}_e(x,v) \leqslant \mathcal{L}_e(0,\mindome)$, 
	\begin{align*}
		a(x,v) = \varphi^{-1}\left(\frac{\frac{v^2}{2} + \varphi(x) - \frac{\mindome^2}{2}}{1+1/\mu}\right)
		\geqslant \frac{\frac{\mindome^2}{2} - (\frac{v^2}{2} + \varphi(x))}{\beta(1+1/\mu)}.
	\end{align*}
	Then 
	\begin{align*}
		f_i(x,v) \leqslant \maxfe \min \left(\frac{1}{\sqrt{|X|}}, \frac{A}{\sqrt{B - X}}\right) 
		\quad \text{with} \quad
		X \coloneqq \frac{v^2}{2} + \varphi(x), \quad A \coloneqq \frac{2}{\sqrt{\frac{\alpha}{\beta(1+1/\mu)}}}, \quad \text{and} \quad B \coloneqq \frac{\mindome^2}{2}.
	\end{align*}
	 The elementary study of the function $X \to \min\left(|X|^{-1/2}, A (B-X)^{-1/2}\right)$ reveals a global maximum at $X = \frac{B}{A^2 + 1} < B$, and we conclude to the result.
	 
 	\begin{figure}
 		\centering
 		\includegraphics[width=0.5\linewidth]{images/global_bound_fi}
 		\caption{Notations for the boundedness of $f_i$.}
 		\mysubcaption{The coloured area corresponds to $\mathcal{L}_e(x,v) \geqslant \mathcal{L}_e(0,\mindome)$. The function $a(x,v)$ gives the point of the ion characteristic (in brown) where $f_e$ vanishes.}
 		\label{fig:charmaps_domainmap}
 	\end{figure}
}

\subsection{Continuity estimates}

\subsubsection{Electron density $n_e$}

\begin{lem}
	The electronic density $n_e$ is continuous.
\end{lem}

\myproof{
	We already know that $n_e$ is bounded. Moreover, using the symmetry $f_e(x,v) = f_e(x,-v)$, we may write
	\begin{align*}
		n_e(x) - n_e(y) = 2 \underbrace{\int_{v=0}^{v_x(y)} f_e(x,v) dv}_{\eqqcolon\,\mathcal{I}^+} + 2 \left(\underbrace{\int_{v=v_x(y)}^{v_x(1)}  f_e(x,v) dv - \int_{v=0}^{v_y(1)} f_e(y,v) dv}_{\eqqcolon\,\mathcal{I}^-}\right).
	\end{align*}
	The term $\mathcal{I}^+$ is bounded by $\maxfe v_x(y) = \maxfe \left(\frac{2}{\mu}\left(\varphi(x) - \varphi(y)\right)\right)^{1/2}$, and by continuity of $\varphi$, we have $\mathcal{I}^{+} \underset{y\to x}{\longrightarrow} 0$. On the first integral of $\mathcal{I}^{-}$, we apply the change of variable
	\begin{align*}
		w = \left(v^2 - \frac{2}{\mu} \left(\varphi(x) - \varphi(y)\right)\right)^{1/2} \quad \iff \quad dv = \frac{w}{\left(w^2 + \frac{2}{\mu}\left(\varphi(x) - \varphi(y)\right)\right)^{1/2}} \,dw, \quad w \in [0,v_y(1)]
	\end{align*}
	to get
	\begin{align*}
		\mathcal{I}^- = \int_{w=0}^{v_y(1)} f_e\left(x,\left(w^2 + \frac{2}{\mu}\left(\varphi(x) - \varphi(y)\right)\right)^{1/2}\right) \frac{w}{\left(w^2 + \frac{2}{\mu}\left(\varphi(x) - \varphi(y)\right)\right)^{1/2}} dw - \int_{v=0}^{v_y(1)} f_e(y,v) dv.
	\end{align*}
	Notice that
	\begin{align*}
		f_e\left(x,\left(w^2 + \frac{2}{\mu}\left(\varphi(x) - \varphi(y)\right)\right)^{1/2}\right)
		= f_{e,b} \left(\frac{w^2}{2} + \frac{1}{\mu}\left(\varphi(x) - \varphi(y)\right) - \frac{1}{\mu} \varphi(x)\right)
		= f_e(y,w).
	\end{align*}
	Renaming $w$ in $v$, we obtain the (clearly nonpositive) expression
	\begin{align*}
		\mathcal{I}^- 
		&= \int_{v=0}^{v_y(1)} f_e(y,v) \left(\frac{v}{\left(v^2 + \frac{2}{\mu}\left(\varphi(x) - \varphi(y)\right)\right)^{1/2}} - 1\right) dv 
		\geqslant \maxfe \int_{v=0}^{v_y(1)}\left(\frac{v}{\left(v^2 + \frac{2}{\mu}\left(\varphi(x) - \varphi(y)\right)\right)^{1/2}} - 1\right) dv \\
		&= \maxfe \left(\left(v_y(1)^2 + \frac{2}{\mu}\left(\underbrace{\varphi(x) - \varphi(y)}_{\geqslant 0}\right)\right)^{1/2} - \left(\frac{2}{\mu}\left(\varphi(x) - \varphi(y)\right)\right)^{1/2} - v_y(1)\right) 
		\geqslant - \maxfe \left(\frac{2}{\mu}\left(\varphi(x) - \varphi(y)\right)\right)^{1/2}
	\end{align*}
	and this shows that $\mathcal{I}^{-} \underset{y\to x}{\longrightarrow} 0$.
}

\begin{lem}\label{lem:ne_continuous_phi}
	Let $n_e^{\varphi}$ and $n_e^{\psi}$ be the electron densities generated by potentials $\varphi$ and $\psi$ satisfying the assumptions. Then
	\begin{enumerate}
	\item If $f_{e,b}$ is Lipschitz-continuous with constant $\lipfe$, then 
	\begin{align*}
		|n_e^{\varphi}(x) - n_e^{\psi}(x)| \leqslant 2 \lipfe \maxdome \sqrt{\frac{2}{\mu}}\left|\varphi(x) -\psi(x)\right|^{1/2} \quad \forall (x,y) \in [0,1]^2.
	\end{align*}
	\item If there exists a constant $[f_{e,b}]$ such that $|f_{e,b}(x) - f_{e,b}(y)| \leqslant [f_{e,b}] |x^2 - y^2|$ \intern{(or equivalently, $x \to f_{e,b}(\sqrt{x})$ is a lipschitz function, as for instance $e^{-x^2}$)}, then 
	\begin{align*}
		|n_e^{\varphi}(x) - n_e^{\psi}(x)| \leqslant \frac{4\maxdome \lipfesq}{\mu} \left|\varphi(x) - \psi(x)\right| \quad \forall (x,y) \in [0,1]^2.
	\end{align*}
	\end{enumerate}
\end{lem}

\myproof{
	We have
	\begin{align*}
		\left|n_e^{\varphi}(x) - n_e^{\psi}(x)\right| 
		\leqslant 2 \int_{v=0}^{\maxdome} \left|f_e^{\varphi}(x,v) - f_e^{\psi}(x,v)\right| dv
		= 2 \int_{v=0}^{\maxdome} \left|f_{e,b}\left(\left(v^2 - \frac{2}{\mu}\varphi(x)\right)^{1/2}\right) - f_{e,b}\left(\left(v^2 - \frac{2}{\mu}\psi(x)\right)^{1/2}\right)\right| dv.
	\end{align*}
	Then, if $f_{e,b}$ is Lipschitz with constant $\lipfe$, we obtain
	\begin{align*}
		\left|n_e^{\varphi}(x) - n_e^{\psi}(x)\right| \leqslant 2 \lipfe \int_{v=0}^{\maxdome} \left|\left(v^2 - \frac{2}{\mu}\varphi(x)\right)^{1/2} - \left(v^2 - \frac{2}{\mu}\psi(x)\right)^{1/2}\right| dv.
	\end{align*} 
	Using \cref{lem:maj_dist_square} yields
	\begin{align*}
		\left|n_e^{\varphi}(x) - n_e^{\psi}(x)\right| 
		\leqslant 2 \lipfe \int_{v=0}^{\maxdome} \left|- \frac{2}{\mu}\varphi(x) + \frac{2}{\mu}\psi(x)\right|^{1/2} dv
		= 2 \lipfe \maxdome \sqrt{\frac{2}{\mu}}\left|\varphi(x) -\psi(x)\right|^{1/2}.
	\end{align*}
	If $f_{e,b}(\sqrt{\cdot})$ is Lipschitz with constant $\lipfesq$, we may directly write
	\begin{align*}
		\left|n_e^{\varphi}(x) - n_e^{\psi}(x)\right| \leqslant 2 \maxdome \lipfesq \frac{2}{\mu} \left|\varphi(x) - \psi(x)\right|.
	\end{align*} 
}


\subsubsection{Ion density $n_i$}

\begin{proposition}
	The density $n_i$ is continuous. 
\end{proposition}

\myproof{
	Let $0 \leqslant y < x \leqslant 1$. For convenience, we represent $n_i(x)$ (resp. $n_i(y)$) as an integral with the artificial lower bound $-g_x(-\maxdome) \leqslant -\maxdome$ (resp. $-g_y(-\maxdome)$). Then
	\begin{align}\label{eq:def_ni_sym}
		n_i(x) - n_i(y) 
		&= 2 \int_{v=-g_x(-\maxdome)}^0 f_i(x_b(x,v),v_b(x,v)) dv - 2 \int_{v=-g_y(-\maxdome)}^0 f_i(x_b(y,v),v_b(y,v)) dv \notag\\
		&= 2 \underbrace{\left[\int_{v=-g_x(-\maxdome)}^{-g_x(y)} f_i(x_b(x,v),v_b(x,v)) dv - \int_{v=-g_y(-\maxdome)}^{0} f_i(x_b(y,v),v_b(y,v)) dv\right]}_{\eqqcolon\,\mathcal{I}^-} \\
		&\quad + 2 \underbrace{\int_{v=-g_x(y)}^{0} f_i(x_b(x,v),v_b(x,v)) dv}_{\eqqcolon\,\mathcal{I}^+}. \notag
	\end{align}

	The term $\mathcal{I}^+$ is clearly nonnegative, and may be adressed using our lemmas. Indeed, using the integral representation of $f_i(x_b,v_b)$ and the reparametrization by a space variable $z$, we have
	\begin{align*}
		\mathcal{I}^+ 
		&= \int_{v=-g_x(y)}^{0} \left[\int_{z=x_b(x,v)}^{x} + \int_{z=x}^{1}\right] \frac{f_e(z,-\left(v^2+2\left(\varphi(x)-\varphi(z)\right)\right)^{1/2}))}{\left(v^2+2\left(\varphi(x)-\varphi(z)\right)\right)^{1/2}} \,dz dv \\
		&\leqslant \maxfe \int_{v=-g_x(y)}^{0} \int_{z=x_b(x,v)}^{x} \frac{1}{\left(v^2-g_x^2(z)\right)^{1/2}} \, dz dv + \maxfe \int_{v=-g_x(y)}^{0} \int_{z=x}^{1} \frac{1}{\left(v^2+2\left(\varphi(x)-\varphi(z)\right)\right)^{1/2}} \,dz dv \\
		&\leqslant \maxfe \left(2\sqrt{\frac{2}{\alpha}} \left(\varphi(y)-\varphi(x)\right)^{1/4} \sqrt{x-y} + \frac{2\sqrt{2}}{\alpha^{1/4}} \left(2(\varphi(y)-\varphi(x))\right)^{1/2}\right),
	\end{align*}
	where we used \cref{lem:upperbound_ni_endchar} for the first term, and \cref{lem:upperbound_ni_beginchar} for the second term (with $y=x$ and $v_0 = -g_x(y)$ under the notations of the lemma). Since $\varphi$ is continuous, we deduce that $\mathcal{I}^{+} \underset{y\to x}{\longrightarrow} 0$. Taking the extreme case $y=0$ and $x=1$, we obtain that 
	\begin{align*}
		\mathcal{I}^+ \leqslant \maxfe \left(2\sqrt{\frac{2}{\alpha}} \left(-\varphi(1)\right)^{1/4} + \frac{2\sqrt{2}}{\alpha^{1/4}} \left(-2\varphi(1)\right)^{1/2}\right) \eqqcolon K.
	\end{align*}

	Let us now focus on $\mathcal{I}^-$. On the first integral, we make the change of variable
	\begin{align*}
		w = -\left(v^2 + 2 \left(\varphi(x) - \varphi(y)\right)\right)^{1/2} \quad v = -\left(w^2 + 2 \left(\varphi(y) - \varphi(x)\right)\right)^{1/2}.
	\end{align*}
	Since $\mathcal{L}_i(x,v) = \mathcal{L}_i(y,w)$, this yields $x_b(x,v) = x_b(y,w)$ and $v_b(x,v) = v_b(y,w)$. The bounds $v\in[-g_x(-\maxdome),-g_x(y)]$ are exactly transported to $w\in[-g_y(-\maxdome),0]$. Renaming $w$ in $v$, we get
	\begin{align*}
		\mathcal{I}^- = \int_{v=-g_y(-\maxdome)}^{0} f_i(x_b(y,v),v_b(y,v)) \left(\frac{-v}{\left(v^2 + 2 \left(\varphi(y) - \varphi(x)\right)\right)^{1/2}} - 1\right) dv.
	\end{align*}
	Since $\varphi(y) \geqslant \varphi(x)$, the factor of $f_i$ is nonpositive, and so is $\mathcal{I}^-$. Moreover, 
	\begin{align*}
		n_i(x) - n_i(y) = 2 \mathcal{I}^- + 2\mathcal{I}^+ \leqslant 2 \mathcal{I}^{-} + 2 K \quad\iff\quad \mathcal{I}^- = - K + n_i(x) - n_i(y) \geqslant - K - |n_i|_{\infty}
	\end{align*}
	and $\mathcal{I}^-$ is bounded. The function $\frac{-v}{\left(v^2 + 2 \left(\varphi(y) - \varphi(x)\right)\right)^{1/2}} - 1$ converges pointwise to 0 when $x\to y$ \todo{and $f_i$ is almost everywhere finite}, and by Lebesgue's dominated convergence, $\mathcal{I}^- \underset{x\to y}{\longrightarrow} 0$. Then $n_i$ is continuous.
}

\section{Results}

Let $0 < \alpha \leqslant \beta$, and define the following convex set:
\begin{align*}
	\K \coloneqq \left\{\varphi\in\mathcal{C}^1([0,1],\mathbb{R}) \quad |\quad \varphi(0)=\varphi'(0)=0, \quad \varphi\ \alpha-\text{concave et } (-\beta)-\text{convex.}\right\}.
\end{align*}
Both variation conditions rewrite
\begin{align*}
	\forall (x,y,\gamma) \in [0,1]^3, \quad 
	\begin{cases}
	 	\varphi((1-\gamma) x + \gamma y) \geqslant (1-\gamma) \varphi(x) + \gamma \varphi(y) + \frac{\alpha}{2} \gamma (1-\gamma) |x-y|^2, \\
	 	\varphi((1-\gamma) x + \gamma y) \leqslant (1-\gamma) \varphi(x) + \gamma \varphi(y) + \frac{\beta}{2} \gamma (1-\gamma) |x-y|^2.
	\end{cases}
\end{align*}
In the case $\varphi\in\mathcal{C}^2$, it is equivalent to $\varphi'' \in [-\beta,-\alpha]$.

\begin{lem}
	$\K$ is closed for the topology induced by the sup-norm $|\cdot|_{\infty} \coloneqq \max_{[0,1]} |\cdot|$.
\end{lem}

\myproof{
	Since $\K \subset \mathcal{C}^{0}([0,1],\mathbb{R})$, we know that any Cauchy sequence $(\varphi_n)_n \subset \K$ admits a limit $\varphi \in \mathcal{C}([0,1],\mathbb{R})$. The pointwise condition $\varphi_n(0)=0$ and the pointwise grows conditions are preserved when $n\to\infty$. Since the family of continuous fonctions $(\varphi'_n)_n$ is equilipschitz, we may use Arzel�-Ascoli to extract an uniformly converging subsequence $\varphi'_{n_k} \to \varphi'_{\infty}$. Using that for all $(x,y)\in[0,1]$,
	\begin{align*}
		\left|\varphi(y)-\varphi(x) - \int_{x}^y \varphi'_{\infty}(z)dz\right| \leqslant |\varphi(y)-\varphi_n(y)| + |\varphi(x)-\varphi_n(x)| + \int_{x}^y |\varphi'_{\infty}(z)-\varphi'_n(z)| dz \underset{n\to\infty}{\longrightarrow} 0,
	\end{align*} 
	we get that $\varphi'_{\infty}$ is the (continuous) derivative of $\varphi$. Finally, the uniform convergence $\varphi'_{n_k} \to \varphi'_{\infty}$ gives $\varphi'(0)=0$.
}

We define an operator $F : \K \mapsto \mathcal{C}^2([0,1],\mathbb{R})$ by the solution of the Poisson problem 
\begin{align*}
	- \lambda^2 F'' = n_i[\varphi] - n_e[\varphi],
\end{align*}
where $n_i$ and $n_e$ are the ion and electron densities obtained with the characteristics induced by $\varphi$. 

\paragraph{Stability}
The function $F$ is solution to a Poisson problem with bounded continuous source term, so it enjoys $\mathcal{C}^2$ regularity in space. The boundary condition $F_{\varphi}(0)=F_{\varphi}'(0)=0$ are satisfied by construction. We turn to the stability of the variation estimates: owing to the boundedness of $n_i$ and $n_e$ stated above, we have
\begin{align*}
	\begin{cases}
		- F''(x) \geqslant \frac{\kappa_1}{\sqrt{\beta} \left(\kappa_2+ \beta\right)^{1/2}}  - \kappa_3, \\
		- F''(x) \leqslant \frac{\kappa_4}{\alpha^{1/4}} \left(\beta^{1/4} + \kappa_5\right)
	\end{cases}
\end{align*}
where
\begin{align*}
	\kappa_1 \coloneqq \frac{\nu \sqrt{2} \minfe (\maxsuppe^2 - \minsuppe^2) \minsuppe}{\lambda^2\maxsuppe \sqrt{1+\frac{1}{\mu}}}, \quad 
	\kappa_2 \coloneqq \frac{\maxsuppe^2}{2}, \quad
	\kappa_3 \coloneqq \frac{2 \maxfe \maxdome}{\lambda^2}, \quad
	\kappa_4 \coloneqq \frac{4\nu (1+\sqrt{2})}{\lambda^2}, \quad
	\kappa_5 \coloneqq 4 \sqrt{2 \maxdome}.
\end{align*}
A sufficient condition for the stability $F(\K) \subset \K$ is to obtain $\alpha \leqslant -F'' \leqslant \beta$. \intern{We may not always have a solution $(\alpha,\beta)$... It depends on the coefficients.}

%\bibliographystyle{alpha}
%\bibliography{CEMRACS.bib}

\end{document}