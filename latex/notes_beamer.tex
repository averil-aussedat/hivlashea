\documentclass{beamer}

%%%%%%%%%%%%%%%%%%%%%%%%%%%%%%%%%%%%%%%%%%%%%%%%%%%%%%%%%%%
% packages 
%%%%%%%%%%%%%%%%%%%%%%%%%%%%%%%%%%%%%%%%%%%%%%%%%%%%%%%%%%%

\usepackage[latin1]{inputenc} 
\usepackage[T1]{fontenc} % hyphenation
\usepackage{xcolor} % color.
\usepackage{theorem}
\usepackage{amssymb,amsmath}  % all you like in math fonts
\usepackage{graphicx}
\usepackage{mathtools} % coloneqq
\usepackage{stmaryrd} % ll/rr brackets
\usepackage[french]{babel} % indentation 
\usepackage{mathrsfs} % \mathscr 
\usepackage{enumitem}
\usepackage{adjustbox}
\usepackage[breakable]{tcolorbox}
\usepackage[tworuled,linesnumbered,vlined,nofillcomment]{algorithm2e}
\usepackage{hyperref} % links & cliquable toc. LAST (but cleverref)
\usepackage[noabbrev]{cleveref} % load after hyperref

%%%%%%%%%%%%%%%%%%%%%%%%%%%%%%%%%%%%%%%%%%%%%%%%%%%%%%%%%%%
% theorems and tcolorboxes
%%%%%%%%%%%%%%%%%%%%%%%%%%%%%%%%%%%%%%%%%%%%%%%%%%%%%%%%%%%

\tcbuselibrary{theorems}
\definecolor{maincolor}{RGB}{10,79,63}

\newtcbtheorem{theom}{Th�or�me}{colback=white, sharp corners, 
breakable, boxrule=0.3mm, separator sign dash, fonttitle=\scshape, 
coltitle=black, detach title, before upper={\textbf{\tcbtitle}\quad}}{thm}

\newtcbtheorem{prop}{Proposition}{colback=white, 
breakable, boxrule=0.4mm, separator sign dash, 
coltitle=black, detach title, before upper={\textbf{\tcbtitle}\quad}}{prop}

\newtcbtheorem{algo}{Algorithme}{colback=white!95!red, fontupper=\ttfamily,
breakable, boxrule=0.2mm, colframe=red!50!black, separator sign dash, fonttitle=\sffamily,
coltitle=black, detach title, before upper={\textbf{\tcbtitle}\vspace{3pt}\\}}{algo}

%\newtcbtheorem{defin}{D�finition}{colback=white!85!maincolor, sharp corners, 
%breakable, boxrule=-1mm, separator sign dash, boxsep=2mm,
%coltitle=black, detach title, before upper={\textbf{\tcbtitle}\quad}}{def}

\newtcbtheorem{defin}{}{colback=white!55!maincolor, sharp corners, 
breakable, boxrule=-1mm, separator sign none, boxsep=2mm,
coltitle=black, detach title, before upper={\hspace{-5pt}\textbf{\tcbtitle}\quad}}{def}
%
%\newtcbtheorem{demo}{D�monstration}{colback=white, sharp corners, coltitle=black,
%breakable, boxrule=-1mm, leftrule=0.4mm, separator sign dash, fonttitle=\normalsize, 
%detach title, before upper={\textbf{\tcbtitle}\vspace{5pt}\\}, after upper={\hfill\large$\square$},
%fontupper=\small}{demo*}
%
\newtcbtheorem{lem}{Lemme}{colback=white, 
breakable, boxrule=0.4mm, separator sign dash, 
coltitle=black, detach title, before upper={\textbf{\tcbtitle}\quad}}{lem}

\newtcbtheorem{rem}{Remarque}{
boxrule=-1mm, shrink tight, separator sign dash, detach title, 
colback=white, coltitle=black, before upper={\textbf{\tcbtitle}\quad}
}{rem}

\newtcbtheorem[crefname={}{}]{hyp}{}{
boxrule=-1mm, lefthand width=0cm,
detach title, theorem name, separator sign none, sharp corners,
colback=white!70!yellow, coltitle=black, before upper={\hspace{-5pt}\textbf{\tcbtitle}\quad}
}{hyp}

%%%%%%%%%%%%%%%%%%%%%%%%%%%%%%%%%%%%%%%%%%%%%%%%%%%%%%%%%%%
% misc 
%%%%%%%%%%%%%%%%%%%%%%%%%%%%%%%%%%%%%%%%%%%%%%%%%%%%%%%%%%%

\definecolor{maincolor}{rgb}{0.5,0.8,0.7}

\usetheme{Copenhagen}
\usefonttheme{professionalfonts}

% num�rotation
\setbeamertemplate{page number in head/foot}[totalframenumber]
\setbeamerfont{page number in head/foot}{size=\tiny}
\setbeamertemplate{section in toc}{\inserttocsection} % no number nor sign
\setbeamertemplate{subsection in toc}[circle] % idem, but for unknown reasons

\beamertemplatenavigationsymbolsempty

%% add toc frame
\AtBeginSection[]
{
	\begin{frame}
		\frametitle{Table of Contents}
		\tableofcontents[currentsection]
	\end{frame}
}

\useoutertheme[subsection=false]{miniframes}

\title{Hamilton-Jacobi via les r�seaux de neurones}
\subtitle {Soutenance de stage ing�nieur}
\author[A. Prost]{\vspace*{25pt}\\ Averil PROST \\ \emph{sous la direction de } \\ Olivier BOKANOWSKI}
\date{}

%%%%%%%%%%%%%%%%%%%%%%%%%%%%%%%%%%%%%%%%%%%%%%%%%%%%%%%%%%%
% commands
%%%%%%%%%%%%%%%%%%%%%%%%%%%%%%%%%%%%%%%%%%%%%%%%%%%%%%%%%%%

\newcommand{\vv}[1]{\begin{pmatrix} #1 \end{pmatrix}} % raccourci pour vecteur
\newcommand{\rot}{\,\textbf{\mbox{rot}}\,} % rotationnel
\newcommand{\tab}{\hspace*{10mm}} % pour le code~: tabulation
\newcommand{\com}[1]{{\color{OliveGreen}\textit{// #1}}} % pour le code~: commentaire
\newcommand{\mypar}[1]{\vspace{4pt}\noindent\textbf{#1}\hspace{6pt}} % my paragraph 
\newcommand{\etape}[1]{\noindent\paragraph{�tape \theetapectr\ (\textbf{#1})\\}\refstepcounter{etapectr}} 
\newcommand{\vs}[1]{{\left[#1\right]^n}} % vector space
\newcommand{\imh}{0.3\textwidth} %image height, meant to be locally adapted
\DeclareMathOperator*{\argmin}{argmin}% argument du minimum

\renewcommand{\div}{\,\mbox{div}\,} % divergence
\renewcommand{\textbf}[1]{\ifmmode{\boldsymbol{#1}}\else{\bfseries\boldmath #1}\fi} % quite dangerous~: pour mettre les maths en gras

\setlist[itemize,1]{label=\color{maincolor}\textbullet}
\setlist[itemize,2]{label=\color{maincolor}\textbullet}
\setlist[itemize,3]{label=\color{maincolor}\textbullet}

%%%%%%%%%%%%%%%%%%%%%%%%%%%%%%%%%%%%%%%%%%%%%%%%%%%%%%%%%%%%%%%%%%%%%%%%%%%%%%%%%%%%%%%%%%
%%%%%%%%%%%%%%%%%%%%%%%%%%%%%%%%%%%%%%%%%%%%%%%%%%%%%%%%%%%%%%%%%%%%%%%%%%%%%%%%%%%%%%%%%%
%%%%%%%%%%%%%%%%%%%%%%%%%%%%%%%%%%%%%%%%%%%%%%%%%%%%%%%%%%%%%%%%%%%%%%%%%%%%%%%%%%%%%%%%%%
%%%%%%%%%%%%%%%%%%%%%%%%%%%%%%%%%%%%%%%%%%%%%%%%%%%%%%%%%%%%%%%%%%%%%%%%%%%%%%%%%%%%%%%%%%

\begin{document}

	\section{Algorithms for the full model}
	
	\begin{frame}{Finite difference discretization}
		Explicit upwind scheme for $f_{e,i}$: 
		\begin{align*}
			\frac{f_{s,k,l}^{n+1} - f_{s,k,l}^{n}}{\Delta t} + \vv{v_l\\c_s E_k^n}_{+} D^-_{k,l} f_s^n + \vv{v_l\\c_s E_k^n}_{-} D^+_{k,l} f_s^n = S_{s,k,l}^n
		\end{align*}
		with $S_e = 0$, $S_i = \nu f_e$, and $D^{\pm}_{k,l} f \coloneqq \pm \left(\frac{f_{k\pm1,l} - f_{k,l}}{\Delta x}, \frac{f_{k,l\pm1} - f_{k,l}}{\Delta v}\right)^t$.
		\pause
		Integration for the electric field $E$, assuming $E(0)=0$:
		\begin{align*}
			E_k^{n+1} = \frac{1}{\lambda^2} \sum_{\kappa} \frac{1}{\Delta x} \left(n_{i,\kappa}^{n} - n_{e,\kappa}^{n}\right)
		\end{align*}
		\begin{itemize}
		\item\pause CFL condition to ensure the stability of the explicit scheme.
		\item\pause First-order approximation, diffusive.
		\end{itemize}		
	\end{frame}
	
	\begin{frame}{Semi-Lagrangian scheme}
		Strang splitting:
		\begin{align*}
			\onslide<.-> {
			\frac{\Delta t}{2} &&
			\begin{cases}
				\partial_t f_s + v \partial_x f_s = 0 & \text{Linear advection at constant speed} \\
				\lambda^2 \partial_x E = n_i - n_e & \text{Resolution by (numerical) integration} \\
				\partial_t f_i = \nu f_e & \text{Pointwise ODE}
			\end{cases} \\
			}
			\onslide<+(1)-> {
			\Delta t &&
			\partial_t f_s + c_s E_s \partial_v f_s = 0 \quad \text{Again, advection at constant speed} \\
			}
			\onslide<+(1)-> {
			\frac{\Delta t}{2} &&
			\begin{cases}
				\partial_t f_i = \nu f_e & \text{Pointwise ODE} \\
				\lambda^2 \partial_x E = n_i - n_e & \text{Resolution by (numerical) integration}\\
				\partial_t f_s + v \partial_x f_s = 0 & \text{Linear advection at constant speed.}
			\end{cases} 
			}
		\end{align*}
		\begin{itemize}
		\item\pause The analytical solution to each step is known.
		\item\pause Use of the 1D solver to enforce boundary conditions.
		\end{itemize}
	\end{frame}
	
	\begin{frame}{Fixed-point algorithm}
		Equilibrium state:
		\begin{align*}
			\begin{cases}
				v \partial_x f_s - c_s \partial_x \varphi_s \partial_v f_s = S_s, \quad \frac{d}{d\tau} [f_s(x_s(\tau),v_s(\tau))] = S_s(x_s(\tau),v_s(\tau)) \\
				-\lambda^2 \partial^2_{xx} \varphi = n_i - n_e
			\end{cases}
		\end{align*}
		\pause
		Suppose $\varphi^k$ is known. Iteration of
		\begin{eqnarray}
			f_e^{k+1}(x,v) &\coloneqq& f_{e,b}(x_e(-\tau),v_e(-\tau)) \nonumber \\ \pause
			f_i^{k+1}(x,v) &\coloneqq& f_{i,b}(x_i(-\tau),v_i(-\tau)) + \int_{-\tau}^0 \nu f_e^{k+1} (x_i(s),v_i(s)) ds \nonumber \\ \pause
			-\lambda^2 \partial^2_{xx} \varphi^{k+1} &\coloneqq& \int_v [f_i^{k+1}(\cdot,v) - f_e^{k+1}(\cdot,v)] dv \nonumber
		\end{eqnarray}
		
	\end{frame}
	
	\begin{frame}
		Merci.
	\end{frame}
	
	
%	    \begin{eqnarray}\label{Poisson}
%	        \left\{
%	        \begin{array}{rll}
%	            -\lambda^2\partial_{xx}^2\phi(x)&=\rho(x) & \text{for $x\in[-1,1]$}, \\
%	            \phi(0)&=0, & \\
%	            \partial_x\phi(0)&=f\in\R. &
%	        \end{array}
%	        \right.
%	    \end{eqnarray}
%	    \pause
%	    \begin{itemize}
%	        \item $\blue{E(t,x):=-\partial_x\phi(t,x)}$ the electric field in \eqref{VlaPoi}.
%	        \end{itemize}
%	    \begin{eqnarray}\label{PoissonE}
%	        \Rightarrow\quad\left\{
%	        \begin{array}{rll}
%	            \lambda^2\partial_x E(x)&=\rho(x) & \text{for $x\in[-1,1]$}, \\
%	            -E(0)&=f\in\R. &
%	        \end{array}
%	        \right.
%	    \end{eqnarray}
%	    \begin{itemize}
%	        \item For all $x\in[-1,1]$
%	    \end{itemize}
%	    \begin{eqnarray}\label{E}
%	        \red{E(x)=\frac{1}{\lambda^2}\int_0^x\rho(y)dy}.
%	    \end{eqnarray}
%	\end{frame}
	
\end{document}