\documentclass[12pt,english]{article}

\usepackage{cmap}
\usepackage[T1]{fontenc}
\usepackage[utf8]{inputenc}
\usepackage{geometry}
\geometry{verbose,tmargin=2.5cm,bmargin=2.5cm,lmargin=2.5cm}
\usepackage{amsmath}
\usepackage{amssymb}
\usepackage{amsthm}
\usepackage{authblk}
\usepackage{physics}
\usepackage{graphicx}
\usepackage{float}
\usepackage{setspace}
\usepackage[normalem]{ulem}

\usepackage{placeins} 


\usepackage{color}
\usepackage{xcolor}
\usepackage{tikz}

\usepackage{hyperref}
\usepackage{biblatex}
\bibliography{biblio}

\theoremstyle{plain}\newtheorem{theorem}{Theorem}[section]
\theoremstyle{plain}\newtheorem{corollary}{Corollary}[section]
\theoremstyle{plain}\newtheorem{lemma}{Lemma}[section]
\newtheorem{prop}{Proposition}[section]
\newtheorem{remark}{Remark}[section]
\newtheorem{property}{Property}[section]
\DeclareSymbolFont{largesymbol}{OMX}{yhex}{m}{n}
\DeclareMathAccent{\Widehat}{\mathord}{largesymbol}{"62}
\newcommand{\alertred}[1]{{\color[rgb]{0.95,0,0.05}{#1}}}
\newcommand{\sumli}{\sum\limits}
\newcommand{\tl}{\tilde}
\newcommand{\vp}{\varphi}
\newcommand{\pa}{\partial}

\newcommand{\aexp}{\mathfrak{exp}}
\newcommand{\aerr}{\mathfrak{e}}

\usepackage{babel}
\usepackage{subfig}
\makeatother
\DeclareMathSizes{14}{14}{9.8}{7}
\graphicspath{{Figure/}}
\begin{document}

\title{HIVLASHEA: High order methods for Vlasov-Poisson model for sheath}
\date{}
\maketitle


% ===================================================================


\section{Introduction}
% ===================================================================

We consider the following model for $(f_e, f_i)(t, x, v)\in \mathbb{R}^+$ and $\phi(t, x)\in \mathbb{R}$, 
$t\geq 0, x\in [-1, 1], v\in \mathbb{R}$ 
\begin{equation}
\label{model}
  \begin{aligned}
   \partial_t f_i + v\partial_x f_i -\partial_x \phi \partial_v f_i = \nu f_e, \;\;\; (\mbox{Vlasov for ions})\\ 
   \partial_t f_e + v\partial_x f_e +\frac{1}{\mu}\partial_x \phi \partial_v f_e = 0, \;\;\; (\mbox{Vlasov for electrons})\\
   -\lambda^2\partial_x^2 \phi = \rho_i-\rho_e, \;\; \rho_{i,e}= \int_\mathbb{R} f_{i, e} dv,  \;\;\; (\mbox{Poisson})
  \end{aligned}
\end{equation}
with the following boundary and initial conditions 
\begin{equation}
\label{bc}
  \begin{aligned}
    f_{i, e}(t, x=-1, v>0)= 0, \;\; f_{i, e}(t, x=1, v<0) = 0, \\ 
    -\partial_x \phi(t, 1)=E(t, 1)=C_L(t), \;\; -\partial_x \phi(t, -1)=E(t, -1)=C_R(t), \\
     \phi(t, 0)=0 \;\;  (\mbox{reference potential}), \\ 
 f_{i, e}(0, x, v)=f_{i, e}^0(x, v),  \\ 
- \lambda^2\partial_x^2\phi(0, x) = \int_\mathbb{R} f_{i}^0(x, v) dv  - \int_\mathbb{R} f_{ e}^0(x, v) dv, 
  \end{aligned}
\end{equation}
where $f_{i, e}^0$ are given and $C_L(t), C_R(t)$ will be defined in \eqref{ampere_bc_phi} and \eqref{ampere_bc_r}. 


We look for solution that are symmetric with respect to $x=0$: 
$$
f_{i, e}(t, x, v) = f_{i, e}(t, -x, -v) \;\;  \forall x\in [-1, 1], \;\; \forall v\in\mathbb{R}, \;\;\; \phi(t, x) = \phi(t, -x), \;\;  \forall x\in [-1, 1].  
$$
This implies 
$$
 \begin{aligned}
\rho_{i, e}(t, x) = \rho_{i, e}(t, -x), \;\; J_{i,e}(t, x) = -J_{i,e}(t, -x), \\
E(t, x)=-\partial_x\phi(t, x) = \partial_x \phi(t, -x) = -E(t, -x). 
 \end{aligned}
$$
We call $E$ the electric field which derives from the electric potential $\phi$ ($E=-\partial_x \phi$), whereas $\rho_{i,e}$ are the ion and electron densities. 
Moreover, $J_{i,e} = \int_{\mathbb{R}} vf_{i,e} dv$ denote the ion and electron currents. 

% ===================================================================

\paragraph{Looking for $C(t)$\\}
Here we derive the boundary condition for $\phi$. 

First, we derive with respect to time the Poisson equation 
$$
-\partial_t (\lambda^2\partial_x^2 \phi) = \partial_t (\rho_i- \rho_e), 
$$
and considering the difference between the $v$-integration of the Vlasov equations gives 
$$
\partial_t (\rho_i- \rho_e) + \partial_x (J_i-J_e) = \nu \rho_e, 
$$
so that, using $E=-\partial_x \phi$, we get 
$$
\partial_x (\lambda^2\partial_t E + J_i-J_e) = \nu \rho_e, \;\;\; \forall x\in [-1, 1]. 
$$
Integrating now in space leads to 
\begin{equation}
\label{ampere_integ}
\lambda^2\partial_t E(t, 1) + (J_i-J_e)(t, 1) = \lambda^2\partial_t E(t, -1) + (J_i-J_e)(t, -1) +\nu \int_{-1}^1 \rho_e(t, x)dx,  
\end{equation}
and using the symmetries highlighted before ($E(t,x)=-E(t, -x)$ and  $J(t,x)=-J(t, -x)$), it comes 
\begin{equation}
\label{ampere_bc}
\lambda^2\partial_t E(t, 1) + (J_i-J_e)(t, 1)  = \frac{\nu}{2} \int_{-1}^1 \rho_e(t, x)dx, 
\end{equation}
or, in terms of $\phi$ 
\begin{equation}
\label{ampere_bc_phi}
\partial_x \phi(t, 1) =\partial_x \phi(0, 1) +\frac{1}{\lambda^2}\int_0^t (J_i-J_e)(s, 1)ds - \frac{\nu}{2\lambda^2}\int_0^t  \int_{-1}^1 \rho_e(s, x)dx ds \equiv C(t). 
\end{equation}
For the condition $x=-1$, \eqref{ampere_integ} gives 
\begin{equation}
\label{ampere_bc_r}
\lambda^2\partial_t E(t, -1) = J(t, 1) - \frac{\nu}{2}\int_{-1}^1 \rho_e(t, y) dy. 
\end{equation}

Our goal is to propose high order numerical methods to approximate the solution of \eqref{model}-\eqref{bc}-\eqref{ampere_bc_phi}. 
As a starting point, we will use several techniques proposed and explained in \cite{badsi}. Below are some areas of works 
that will be explored and incorporated in the code developed in \cite{badsi}:  
\begin{itemize}
\item to reach high order at the boundaries, we will use \cite{boutin, coulombel} (code dedicated to the linear transport to test 
with the semi-Lagrangian method). 
\item Hermite-Lagrange 1D interpolation  (see Section \ref{hemite}) (code linear transport and then 
\begin{itemize}
\item code for linear transport (semi-Lagrangian method and periodic boundary conditions) 
\item code for Vlasov-Poisson (one species and periodic boundary conditions)
\item code for Vlasov-Poisson (two species and periodic boundary conditions: test on exact solution from \cite{morel}) 
\end{itemize}
\item write a solver for the Poisson equation with Dirichlet-Neuman boundary conditions. 
\item investigate high order time splittings for \eqref{model}-\eqref{bc}-\eqref{ampere_bc_phi}.  
\end{itemize}
But also
\begin{itemize}
\item Poisson reformulation following ideas in \cite{degond} to overcome the stringent condition on the time step. 
\item construction of the stationary solutions of  \eqref{model}-\eqref{bc}-\eqref{ampere_bc_phi}.  
\end{itemize}

% ===================================================================

\paragraph{Energy evolution\\}
Multiplying by $v^2/2$ the ion equation and by $\mu v^2/2$ the electron equation gives, after integration in $v$ and $x$
$$
\frac{d}{dt}{\cal E}_{\mbox{kin}} + \int\int_{-1}^1 \frac{v^3}{2}\partial_x (f_i + \mu f_e)dxdv -\int_{-1}^1  E (J_i-J_e) dx = \frac{\nu}{\mu} { \cal E}^e_{\mbox{kin}}, 
$$
with ${\cal E}_{\mbox{kin}} = \int\int_{-1}^1 \frac{v^2}{2}(f_i+\mu f_e) dx dv$, ${ \cal E}^e_{\mbox{kin}}= \int\int_{-1}^1 \frac{v^2}{2} \mu f_e dx dv$ and $J_{i,e} = \int v f_{i,e} dv$. \\
But the second term is 
$$
\int\int_{-1}^1 \frac{v^3}{2}\partial_x (f_i + \mu f_e)dxdv = M_3(t, 1)-M_3(t, -1), 
$$
with $M_3(t, x) := \int \frac{v^3}{2}(f_i(t, x, v) +\mu f_e(t, x, v))dv$. 
Using the Amp\`ere equation 
$$
J(t, x) - J(t, -1)= -\lambda^2\partial_t E(t, x) + \lambda^2\partial_t E(t, -1) + \nu \int_{-1}^x \rho_e (t, y)dy, 
$$
we get for the third term 
\begin{eqnarray*}
\int_{-1}^1 EJ dx &=& \int_{-1}^1 E dx \Big[ J(t, -1)+\lambda^2 \partial_t E(t, -1)\Big] -\lambda^2\int_{-1}^1  E \partial_t E(t, x) dx  
+\nu \int_{-1}^1 E \int_{-1}^x  \rho_e (t, y)dy dx\nonumber\\
&=&  - \frac{\lambda^2}{2} \frac{d}{dt} \int_{-1}^1  E^2 dx - \nu\int_{-1}^1 (\partial_x \phi) \int_{-1}^x  \rho_e (t, y)dy dx\nonumber\\
&=& - \frac{\lambda^2}{2} \frac{d}{dt} \int_{-1}^1  E^2 dx - \nu\Big[ \phi(t, 1)\int_{-1}^1  \rho_e (t, y)dy \Big] 
+ \nu\int_{-1}^1 \phi(t, x) \rho_e(t, x) dx, 
\end{eqnarray*}
since $\int_{-1}^1 E dx = -(\phi(t, 1)-\phi(t, -1)) = 0$. Hence, we get 
\begin{eqnarray*}
\frac{d}{dt}{\cal E}_{\mbox{kin}} \!\!\!\!\!\!\!\!\!&& +  \frac{d}{dt}{\cal E}_{\mbox{pot}} +M_3(t, 1)-M_3(t, -1) + \nu\phi(t, 1) \int_{-1}^1  \rho_e (t, x)dx\nonumber\\
&& = \nu \int_{-1}^1  \phi(t, x)\rho_e (t, x)dx + \frac{\nu}{\mu} { \cal E}^e_{\mbox{kin}}, 
\end{eqnarray*}
with ${\cal E}_{\mbox{pot}} = \frac{\lambda^2}{2}\int_{-1}^1 E^2dx$. 

% ===================================================================

\paragraph{Mass evolution\\}
We integrate in $x,v$ the ion and electron equations. Denoting $m_{i,e}(t)=\int \int_{-1}^1 f_{i,e} dxdv$ leads to 
\begin{eqnarray*}
m'_i &+& 2J_i(t, 1) = \nu m_e\nonumber\\
m'_e&+& 2J_e(t, 1) =0.  
\end{eqnarray*}
Thus the difference between the 2 above equations gives $\frac{d}{dt}(m_i-m_e) + 2J(t, 1) = \nu m_e$. 
Using the expression of $J(t, 1)$ from \eqref{ampere_bc}, we finally obtain 
$$
\frac{d}{dt}(m_i-m_e) - 2\lambda^2 \partial_t E(t, 1) + \nu m_e = \nu m_e, 
$$
or 
$$
\frac{d}{dt}(m_i-m_e - 2\lambda^2  E(t, 1)) = 0. 
$$

% ===================================================================

\section{Time integration}
Here a time splitting is presented. We focus on a second order splitting. 

\paragraph{Space transport ${\cal T}_x^{\Delta t/2}$}
Knowing $f^n_{i,e}, E^n$, we want to compute $f^\star_{i,e}, E^\star$ by computing the solution on $[t^n, t^n+\Delta t/2]$ of 
\begin{eqnarray*}
\partial_t f_{i,e} + v\partial_x f_{i,e} &=& 0\nonumber\\
\lambda^2\partial_x E &=& \rho_i-\rho_e\nonumber\\
\lambda^2\partial_t E(t, \pm 1) &=& -J(t, \pm 1) \mp \frac{\nu}{2} m_e(t),   
\end{eqnarray*}
with $m_e(t)=\int_{-1}^1 \rho_e(t, x)dx$. 

To do so, we first compute $f^\star$ using the semi-Lagrangian method. 
Thus, we compute $\rho_{i,e}^\star, J^\star(\pm 1), m_e^\star$ 
so that we can advance the boundary conditions for $E$ through 
$$
\lambda^2 (E^\star(\pm 1) -E^n(\pm 1)) =\frac{\Delta t}{2}\Big[ -\frac{1}{2}(J^n(\pm 1)+J^\star(\pm 1)) \mp \frac{\nu}{4}(m_e^n+m_e^\star)\Big],   
$$
with a second order accuracy. After this update, one can compute $E^\star$ with the Poisson solver with the right hand side 
$\rho_i^\star-\rho_e^\star$. 

\paragraph{Velocity transport ${\cal T}_v^{\Delta t/2}$}
Knowing $f^n_{i,e}, E^n$, we want to compute $f^\star_{i,e}, E^\star$ by computing the solution on $[t^n, t^n+\Delta t/2]$ of 
\begin{eqnarray*}
\partial_t f_{i} + E\partial_v f_{i} &=& 0\nonumber\\
\partial_t f_{e} -\frac{ E}{\mu}\partial_v f_{e} &=& 0. 
\end{eqnarray*}
Since $E$ does not change in time along this equations, one only has to update $f_{i,e}$ using the semi-Lagrangian method. 

\paragraph{Source term ${\cal S}^{\Delta t}$}
Knowing $f^n_{i,e}, E^n$, we want to compute $f^\star_{i,e}, E^\star$ by computing the solution on $[t^n, t^n+\Delta t]$ of 
\begin{eqnarray*}
\partial_t f_{i} &=& \nu f_e\nonumber\\
\partial_t f_{e} &=& 0. 
\end{eqnarray*}
Along this step, $E, f_e$ do not change and one can update $f_i$ thanks to 
$$
f^{\star}_i = f^n_i + \nu\Delta t f^n_e. 
$$

\paragraph{Splitting}
The time splitting enabling to compute $(f_i^{n+1}, f_e^{n+1}, E^{n+1})$ from $(f_i^{n}, f_e^{n}, E^{n})$ 
can be written as 
$$
(f_i^{n+1}, f_e^{n+1}, E^{n+1}) = ({\cal T}_x^{\Delta t/2} \circ {\cal T}_v^{\Delta t/2}\circ {\cal S}^{\Delta t}\circ {\cal T}_v^{\Delta t/2} \circ  {\cal T}_x^{\Delta t/2} ) (f_i^{n}, f_e^{n}, E^{n})
$$

% ===================================================================

\section{Examples}
\label{examples}
\subsection{Test 1d}
We can consider the following 1d test 
$$
\partial_t + a \partial_x u = 0, \;\; x\in [0, 1], \;\; a>0, 
$$
with the following boundary and initial conditions $u(t=0, x)=u_0(x), u(t, 0)=u_L(t)$. 
We introduce a mesh in space $x_i=i\Delta x, i=0, \dots, N$, with $\Delta x>0$ and $N\Delta x=1$ 
and in time $t^n=n\Delta t, \Delta t>0$.  
To reconstruct using high order interpolation, one needs to give a sense to the numerical solution at the inflow ghost points  $x_{-i}, i=1, 2, \dots$ at the outflow ghost points $x_{N+i}, i=1, 2, \dots$. 

For the inflow boundary, one can investigate two options 
\begin{itemize}
\item characteristics based: using the characteristics, we have $u(t^n, x_{-1}) = u(t, x_{-1}+a(t-t^n)) = u_L(t)$ with $t$ the time such that 
$x_{-1} + a(t-t^n) = 0$, or $t=t^n-x_{-1}/a$. Of course, the same strategy can be applied for $x_{-i}, i=2, 3, ...$.  
\item Inverse Lax-Wendroff (ILW): following the strategy developed in \cite{boutin, coulombel}, we perform a Taylor expansion 
of inflow ghost points at the boundary $x=0$ 
$$
u_{-1}^n \approx \sum_{k=0}^K \frac{(-\Delta x)^k}{k!}(\partial_x^k u^n)_0 = \sum_{k=0}^K \frac{(\Delta x)^k}{a^k k!}(\partial_t^k u^n)_0 
= \sum_{k=0}^K \frac{(\Delta x)^k}{a^k k!}(\partial_t^k u_L(t^n)). 
$$ 
Of course, the same strategy can be applied for $x_{-i}, i=2, 3, ...$.  
\end{itemize}
One can observe and check properly that the ILW is an approximation of the characteristics based approach. The two options can be programmed in combination with a high order semi-Lagrangian method. 

For the outflow, one can use the strategy proposed in \cite{boutin, coulombel} where a condition is imposed to reconstruct 
$u_{N+i}, i=1, 2, \dots$. For a $k_b$ given, they introduce the finite difference operator $(D u)_j = u_j-u_{j-1}$. 
We impose $(D^2 u)_N=0$ which means $u_{N-1}-2u_N+u_{N+1}=0$ or $u_{N+1}=2u_N-u_{N-1}$. 
The next point $u_{N+2}$ is then defined with $(D^2 u)_{N+1}=u_{N}-2u_{N+1}+u_{N+2}=0$. 
Another choice would be $(D^3 u)_N=0$ which imposes $u_{N+1}$ and so on.  

Tests can be performed using the exact solution $u(t, x)=\sin(k(x-at))$ which means 
that the initial condition is  $u_0(x)=\sin(kx)$ and the boundary condition is $u_L(t)=-\sin(kat)$ 
for a given $k\in\mathbb{N}^\star$. 

% ===================================================================

\subsection{Test 2d}
We suggest to investigate the properties of the above approach in a 2d context using the following model 
$$
\partial_t u +v\partial_x u - x\partial_v u = 0, \;\; x\in [-1, 1], \;\; v\in [-1, 1], 
$$
with the initial condition $u(t=0, x, v) = u_0(x, v)$ and with the boundary conditions in space 
$$
u(t, -1, v)=u_L(t, v), v>0, \;\;\; u(t, 1, v)=u_R(t, v), v<0, 
$$
whereas in $v$, vanishing boundary conditions are considered. 

We shall consider the following exact solution 
$$
u(t, x, v) = u_0(x\cos t - v\sin t , x\sin t +v\cos t), 
$$
so that the boundary conditions are $u_L(t) = u_0(-\cos t -v\sin t, -x\sin t +v\cos t)$ for $v> 0$ 
and $u_R(t) = u_0(\cos t -v\sin t, \sin t +v\cos t)$ for $v< 0$. The initial condition can be chosen as 
$u_0(x, v) = \exp( - 100 (x-1)^2 - 100 v^2)$.  

To use a directional splitting method, one has to modify the boundary conditions to be compatible 
with the one dimensional space advections. 
\begin{itemize}
\item solve $\partial_t u +v \partial_x u = 0$ on $\Delta t/2$ 
with $\tilde{u}_L(t^n) = u_0(x_{-1}\cos t^n - v\sin t^n , x_{-1}\sin t^n +v\cos t^n)$
\item solve $\partial_t u -x \partial_v u = 0$ on $\Delta t$ with zero boundary conditions at the boundary 
\item solve $\partial_t u +v \partial_x u = 0$ on $\Delta t/2$ with $\tilde{u}_L(t) = u_0(x_{-1}\cos t - v\sin t , x_{-1}\sin t +v\cos t)$ with 
$t$ such that $v(t-t^{n+1}) = -\Delta x$. 
\end{itemize}

% ===================================================================

\subsection{Poisson solver}
\label{poisson}
To solve the Poisson equation 
$$
\lambda^2\partial_x E = \rho, \;\; x\in [-1, 1], 
$$
with $E(1)$ given, one integrates successively between $x$ and $1$ and then between $-1$ and $x$ to get 
$$
E(1) - E(x) = \frac{1}{\lambda^2} \int_x^1 \rho(y) dy, \;\;\; E(x) - E(-1) = \frac{1}{\lambda^2} \int_{-1}^x \rho(y) dy. 
$$
We obtain the following expression for $E(x)$ 
$$
E(x) = \frac{1}{2}(E(1)+E(-1)) + \frac{1}{2\lambda^2} \Big[\int_{-1}^x \rho(y) dy - \int_x^1 \rho(y) dy \Big]. 
$$
We assume that $\rho$ is known on the spatial grid $x_i=i\Delta x$ so that $\rho(x_i) \approx \rho_i$. 
Hence, an approximation of the electric field at $x_i$ is given by 
$$
E_i = \frac{1}{2}(E(1)+E(-1)) + \frac{\Delta x}{4\lambda^2} \Big[\sum_{j=0}^{i-1} (\rho_j+\rho_{j+1})  - \sum_{j=i}^{N-1}(\rho_j+\rho_{j+1}) \Big].
$$
This expression ensures the symmetry of $E$ at the discrete level.

% ===================================================================

\section{Codes} 
For testing purposes, codes are developed and validated on examples described below. 

\subsection{Two-species Vlasov-Poisson with periodic boundary conditions}  
A code solving the two-species Vlasov-Poisson with periodic boundary conditions is available in Julia and Python. 
In particular, exact stationary solutions are tested. Ask to Pierre Navaro. 

\subsection{Two-species Vlasov-Poisson with boundary conditions: version 1}
The Mehdi's code solves the full model \eqref{model}-\eqref{bc} with an upwind scheme + explicit Euler time integrator + 
Poisson solver described in \ref{poisson}. It gives a reference numerical solution for the sheath development. 
The initial condition is 
$$
f_i^0(x, v) = \frac{1}{\sqrt{2\pi}} e^{-v^2/2}, \;\;\; f_e^0(x, v) = \frac{\sqrt{\mu}}{\sqrt{2\pi}} e^{-\mu v^2/2}, \;\;\; \phi(0, x)=E(0, x)=0. 
$$

\subsection{Two-species Vlasov-Poisson with boundary conditions: version 2}
The goal would be to integrate the numerical methods developed and validated for the problems of Section \ref{examples} 
in the Michel's code.  

% ===================================================================

\section{Hermite interpolation}
\label{hemite}
Let us consider the 3rd order interpolation on the interval $[x_i, x_{i+1}]$ such that $x_{i+1}-x_i=h>0$. 
We look for the polynomial function $P\in \mathbb{R}^2[X]$ such that 
\begin{equation}
\label{hermite_conditions}
P(x_i) = f(x_i), P(x_{i+1}) = f(x_{i+1}), P'(x_i) = f'(x_i), P'(x_{i+1}) = f'(x_{i+1}).  
\end{equation} 
On the one side, we define the two 1st order Lagrange polynomial 
$$
L_i(x)=\frac{x-x_{i+1}}{x_i-x_{i+1}}  =\frac{x_{i+1}-x}{h}  \mbox{ and } L_{i+1}(x)=\frac{x-x_{i}}{x_{i+1}-x_{i}}=\frac{x-x_{i}}{h}, 
$$
and following the Peyr\'e's tweet, the Hermite polynomial satisfying \eqref{hermite_conditions} is 
\begin{equation}
\label{p-hermite}
P(x) = f(x_i)H_i(x) + f(x_{i+1})H_{i+1}(x)  + f'(x_i)K_i(x) + f'(x_{i+1})K_{i+1}(x),  
\end{equation}
with 
\begin{eqnarray}
\label{K_H}
K_i(x)&=&L_i(x)^2(x-x_i), \;\; K_{i+1}(x)=L_{i+1}(x)^2(x-x_{i+1}), \\
H_i(x)&=&L_i(x)^2(1-2L_i'(x_i)(x-x_i)), \;\; H_{i+1}(x)=L_{i+1}(x)^2(1-2L_{i+1}'(x_{i+1})(x-x_{i+1}). 
\end{eqnarray}
From the properties of $H_i, K_i$, we can check that \eqref{p-hermite} satisfies \eqref{hermite_conditions}. 

We can generalize to arbitrary (odd) order in the following way. 
From a Lagrange polynomial of degree $(2d+1)$, we can reconstruct using Peyr\'e's tweet 
a polynomial of degree $2(2d+1)+1$. 
In this method, we need to reconstruct $f'(x_i)$ which is not known since we only know $f(x_i)$. 
To do so, we can use finite difference formula with the correct order (to not destroy the order of the Hermite reconstruction). 

The strategy is the following. First we define the Lagrange polynomial of order  $2d+1$ ($d\in\mathbb{N}$ given) 
$$
L_i(x)=\Pi_{j\neq i} \frac{x-x_j}{x_i-x_j}, \;\; \mbox{ for } i=-d, \dots, d+1. 
$$
We also define the derivative of the Lagrange polynomial. We only need $L'_i(x_i)$ for which we have 
$$
L_i'(i) = \sum_{j=-d, ..., d+1 /  j \neq i} \frac{1}{i-j}. 
$$ 
Then, we define $(K_i)_{i=-d, \dots, d+1}$ and $(H_i)_{i=-d, \dots, d+1}$ from \eqref{K_H} and we have the following polynomial 
$$
P(x) = \sum_{i=-d}^d f(x_i) H_i(x) + \sum_{i=-d}^d f'(x_i) K_i(x). 
$$
An approximation of $f'(x_i)$ is required. Different options can be envisaged according to stability. 
The first one is the uncentered one as in \cite{caim} 
$$
P(x) = \sum_{i=-d}^d f(x_i) H_i(x) + \sum_{i=-d}^0 f'_{i, +} K_i(x)+ \sum_{i=1}^d f'_{i, -} K_i(x). 
$$
where the derivatives are approximated with 
\begin{eqnarray*}
f'(x_i) &\approx& f'_{i, +} = \sum_{\ell = r^+}^{s^+} b_\ell^+ f_{i+\ell}   \mbox{ for } i=-d, \dots, 0, \\ 
f'(x_i) &\approx& f'_{1, -} = \sum_{\ell = r^-}^{s^-} b_\ell^- f_{i+\ell}   \mbox{ for } i=1, \dots, d.  
\end{eqnarray*} 
The integers $r^{\pm}, s^\pm$ are computed according to a degree $p\in \mathbb{N}^\star$: 
$r^+=-\left\lfloor\dfrac{p}{2}\right\rfloor, \;\; s^+=\left\lfloor\dfrac{p+1}{2}\right\rfloor, \;\; r^-=-s^+, \;\; s^-=-r^+$. 
Below are the coefficients for different values of $p$ (general formulas can be found in \cite{caim} 
\begin{eqnarray*}
p=1 && b^+ = (-1, 1)\\
p=2 && b^+ = (-1/2, 0, 1/2)\\
p=3 && b^+ = (-1/3,-1/2,1,-1/6)\\
p=4 && b^+ = (1/12,-2/3,0,2/3,-1/12)\\
p=5 && b^+ = (1/20,-1/2,-1/3,1,-1/4,1/30)\\
p=6 && b^+ = (-1/60,3/20,-3/4,0,3/4,-3/20,1/60), 
\end{eqnarray*}
with $b^-_{-\ell} = -b^+_\ell$ for $\ell=r^+, \dots, s^+, \ell\neq 0$ and $b^-_0 = -b_0^+$. 

More details of the method in the maple code of Michel !  
\begin{verbatim}
d:=1:r:=-d:s:=d+1:
dmax:=2*(s-r+1):
for k from r to s do
         \% Lagrange polynomials of odd degree 
        	L[k]:=product((x-j)/(k-j),j=r..k-1)*product((x-j)/(k-j),j=k+1..s): 
        	\% derivative of Lagrange 
        	dL[k]:=diff(L[k],x):
od:	
for k from r to s do 
        \% Compute H and K from the Peyr�\'e formulas  
        	H[k]:=L[k]**2*(1-2*subs(x=k,dL[k])*(x-k)):#f
        	K[k]:=L[k]**2*(x-k):  #fprime
od:
for k from 0 to dmax do
        \%test on a given function 
         f:=x**k:df:=diff(f,x):
        	g:=simplify(x**k-add(H[j]*subs(x=j,f),j=r..s)-add(K[j]*subs(x=j,df),j=r..s)):
        	print(k,g);
od:
\end{verbatim}

\begin{thebibliography}{}

\bibitem{badsi}
{\sc M. Badsi, M. Mehrenberger, L. Navoret}
\textit{Numerical stability of plasma sheath}, 
ESAIM: Proceedings and Surveys, SMAI 2017, Volume 64 (2018), pp 17-36.


\bibitem{boutin}
{\sc B. Boutin, T. H. T. Nguyen, Abraham Sylla, S. Tran-Tien, J.-F. Coulombel}, 
\textit{High order numerical schemes for transport equations on bounded domains}, 
(2021), ESAIM: Proceedings and Surveys.

\bibitem{coulombel}
{\sc J.-F. Coulombel,  F. Lagouti\`ere}. 
\textit{The neumann numerical boundary condition for transport equations}. 
Kinet. Relat. Models,   2020, 13(1), pp. 1-32. 

\bibitem{degond}
{\sc P. Degond, F. Deluzet, L. Navoret, A.-B. Sun, M.-H. Vignal}, 
\textit{Asymptotic-Preserving Particle-In-Cell method for the Vlasov-Poisson system near quasi-neutrality}, 
J. Comp. Phys., vol. 229, 16, pp. 5630-5652 (2010)

\bibitem{caim}
{\sc A. Hamiaz, M. Mehrenberger, H. Sellama, E. Sonnendrucker}, 
\textit{The semi-Lagrangian method on curvilinear grids}, 
Communications in Applied and Industrial Mathematics, 2016, Volume 7, Issue 3, pp. 96-134.


\bibitem{morel}
{\sc C. Morel}, 
\textit{A well-balanced scheme using exact solutions to the two species Vlasov-Poisson system}, 
Intern notes. 

\bibitem{sonnen}
{\sc E. Sonnendrucker}
\textit{Numerical Methods for the Vlasov-Maxwell equations}.  

\end{thebibliography}

\end{document}

\begin{prop}
The system \eqref{model}-\eqref{bc}-\eqref{ampere_1} is equivalent to \eqref{model}-\eqref{bc}-\eqref{ampere}. 
\end{prop}
\begin{proof}
The previous calculations showed that \eqref{model}-\eqref{bc}-\eqref{ampere_1} implies \eqref{model}-\eqref{bc}-\eqref{ampere}. 
Let now start with \eqref{model}-\eqref{bc}-\eqref{ampere}. 

We derive \eqref{ampere} with respect to $x$ to get 
$$
-\lambda^2\partial_x\partial_t E +\partial_x J = \frac{\nu}{2}\Big( n_e(t, x) + n_e(t, x)) = \nu n_e(t, x).  
$$
Thanks to the continuity equation $\partial_t (n_i-n_e) +\partial_x J = \nu n_e$, we obtain 
$$
-\lambda^2\partial_x\partial_t E - \partial_t (n_i-n_e) + \nu n_e = \nu n_e, 
$$
or again 
$$
\partial_t (-\lambda^2\partial_x E -  (n_i-n_e) ) = 0. 
$$
Integrating on time $s\in [0, t]$ enables to get 
$$
-\lambda^2\partial_x E(t, x) -  (n_i-n_e)(t, x) = -\lambda^2\partial_x E(0, x) -  (n_i-n_e)(0, x) =0, 
$$
where the last equality comes from the fact that Poisson is satisfied initially. Then, using $E=-\partial_x \phi$, 
we obtain the Poisson equation $\lambda^2\partial^2_x \phi(t, x) =  (n_i-n_e)(t, x)$. Regarding the boundary 
\end{proof}

\begin{remark}
For stationary solution, thanks to $J(t, L)-J(t, -L) = \nu\int_{-L}^L n_e(t, x)dx$, we recover Eq (71) from Alvaro et al. paper.  
\end{remark}

Objectif: \'ecriture d'un code VP double esp\`ece avec bord ($f_{i,e}$ et $\phi$) selon les calculs pr\'ec\'edents. 
Lid\'ee est d'observer la relaxation vers un \'etat stationnaire non homog\`ene comme dans Alvaro et al. 

Quelques sch\'emas possibles 
\begin{itemize}
\item bas\'e sur Duhamel, on peut lever la CFL de transport (papier avec Dimarco-Vignal)
\item semi-Lagrangien 2d pour suivre au mieux les caract\'eristiques (papier avec Degond-Belaouar) 
\item voir pour rester stable par rapport \`a $\lambda$ 
\item interpolation Hermite (Peyr\'e/Michel)
\item voir calculs Morel pour \'etats stationnaires dans le cas p\'eriodique en $x$ 
\end{itemize}

\end{document}